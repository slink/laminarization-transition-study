\documentclass[11pt]{article}
\usepackage{amsmath,amssymb,graphicx}
\usepackage[margin=1in]{geometry}
\usepackage{booktabs}
\usepackage{subcaption}
\usepackage{xcolor}
\usepackage{url}
\usepackage{natbib}

% ---------------------------------------------------------------------------
% Figure macros (article format: single-column figures)
% ---------------------------------------------------------------------------

\newcommand{\figGridConvergence}{%
  \begin{figure}[htbp]
  \centering
  \includegraphics[width=0.85\textwidth]{figures/grid_convergence.pdf}
  \caption{Grid convergence of cycle-averaged viscosity ratio for six
  representative cases.  The dashed line marks the
  $\langle\nu_t\rangle/\nu = 10$ threshold separating laminar and
  turbulent regimes.  Clearly laminar and turbulent cases converge
  quickly; near-transition cases require $N \geq 128$.}
  \label{fig:grid-convergence}
  \end{figure}
}

\newcommand{\figRegimeDiagram}{%
  \begin{figure}[htbp]
  \centering
  \includegraphics[width=\textwidth]{figures/regime_diagram.pdf}
  \caption{Regime classification in the $(Re, \Lambda)$ plane for each
  settling number $S$.  Red squares: turbulent; blue circles: laminar.
  Data from Phases~1 and~3 (601~converged cases).  Low $S$ (slow settling)
  admits a transition at moderate $\Lambda$ that is often non-monotonic;
  high $S$ (fast settling) confines sediment near the bed, leaving
  turbulence undamped.}
  \label{fig:regime-diagram}
  \end{figure}
}

\newcommand{\figViscositySlices}{%
  \begin{figure}[htbp]
  \centering
  \includegraphics[width=\textwidth]{figures/viscosity_slices.pdf}
  \caption{Cycle- and depth-averaged viscosity ratio versus stratification
  parameter $\Lambda$ for six $(Re, S)$ slices.  Dashed line:
  $\langle\nu_t\rangle/\nu = 10$ threshold.  Points are coloured by
  regime (red: turbulent, blue: laminar).  Non-monotonic behaviour
  (turbulent--laminar--turbulent) is visible at $(Re, S) = (300, 0.005)$
  and $(500, 0.01)$.}
  \label{fig:viscosity-slices}
  \end{figure}
}

\newcommand{\figCriticalLambda}{%
  \begin{figure}[htbp]
  \centering
  \includegraphics[width=0.55\textwidth]{figures/critical_lambda.pdf}
  \caption{First laminar onset $\Lambda_\ell$ versus $Re$ for
  $S = 0.005$ and $S = 0.01$.  The onset shifts to lower $\Lambda$ with
  increasing $Re$, but a single critical value is ill-posed because the
  transition is non-monotonic at most $(Re, S)$.}
  \label{fig:critical-lambda}
  \end{figure}
}

\newcommand{\figProductionDiagnostics}{%
  \begin{figure}[htbp]
  \centering
  \includegraphics[width=\textwidth]{figures/production_diagnostics.pdf}
  \caption{Phase~4 production diagnostics.
  (a)~Drag coefficient $c_f$ and (b)~cycle- and depth-averaged kinetic energy
  $\langle KE \rangle$ versus viscosity ratio
  $\langle\nu_t\rangle/\nu$.  Red squares: turbulent; blue circles:
  laminar.
  Dashed line: regime threshold at $\langle\nu_t\rangle/\nu = 10$.}
  \label{fig:production-diagnostics}
  \end{figure}
}

\newcommand{\figPhasePortraits}{%
  \begin{figure}[htbp]
  \centering
  \includegraphics[width=\textwidth]{figures/phase_portraits.pdf}
  \caption{Phase portraits $(u_{\mathrm{bed}}, \tau_{\mathrm{bed}})$ for
  four representative Phase~4 cases.  Triangles mark the starting point.
  Blue: weakly turbulent ($Re=200$); red: strongly turbulent ($Re \geq 300$).
  Weakly turbulent cases produce tighter orbits with lower
  $\tau_{\mathrm{bed}}$ amplitude than the strongly turbulent cases.}
  \label{fig:phase-portraits}
  \end{figure}
}

\newcommand{\figReentrantProfiles}{%
  \begin{figure}[htbp]
  \centering
  \includegraphics[width=\textwidth]{figures/reentrant_profiles.pdf}
  \caption{Cycle-averaged vertical profiles for three cases spanning the
  re-entrant window at $(Re, S) = (1000, 0.005)$: $\Lambda = 0.1$
  (turbulent, solid red), $\Lambda = 2.0$ (laminar, solid blue), and
  $\Lambda = 5.0$ (re-entrant turbulent, dashed red).
  (a)~Concentration $C(z)$; (b)~gradient Richardson number $Ri_g(z)$,
  with the critical value $Ri_c = 0.25$ shown as a dotted line;
  (c)~eddy viscosity $\nu_t(z)$.  At high $\Lambda$, the sharp
  near-bed concentration gradient reduces outer-layer $Ri_g$ below
  $Ri_c$, restoring turbulence.}
  \label{fig:reentrant_profiles}
  \end{figure}
}

\newcommand{\figDampingComparison}{%
  \begin{figure}[htbp]
  \centering
  \includegraphics[width=\textwidth]{figures/damping_comparison.pdf}
  \caption{Regime classification on the $(\Lambda,\, Re_\delta)$ plane for
  linear damping (left) and exponential damping (right) at
  $S = 0.005$ (top) and $S = 0.01$ (bottom).  Red squares: turbulent;
  blue circles: laminar.  The transition boundary shifts to moderately
  higher~$\Lambda$ with exponential damping, but the qualitative topology
  is preserved.}
  \label{fig:damping_comparison}
  \end{figure}
}

% ---------------------------------------------------------------------------

\title{Laminarization Transition in Sediment-Laden Oscillatory Boundary Layers: A Computational Phase Map}

\author{Shmuel Link\\
\small Independent Researcher, Santa Cruz, California, United States\\
\small \texttt{sglink@gmail.com}}

\begin{document}
\maketitle

\begin{abstract}
Wave-driven sediment transport models assume that turbulence persists in the
oscillatory boundary layer, yet field and laboratory observations show that high
suspended-sediment concentrations can suppress turbulence and trigger a
laminar-like regime through a positive feedback in which stratification damps
turbulent mixing, reducing vertical sediment diffusion and thereby reinforcing
the stratification. We present a systematic computational study of this
transition using a one-dimensional, vertically resolved model coupling momentum
and sediment advection--diffusion with an algebraic stratification-damping
closure. By sweeping the parameter space spanned by the Reynolds number $Re$,
settling number $S$, and stratification parameter $\Lambda$, we produce a regime
phase diagram mapping the transition topology $\Lambda_c(Re, S)$, which
exhibits re-entrant boundaries where turbulence collapses at intermediate
stratification but recovers at higher $\Lambda$. Because this critical surface depends on the
adopted closure---here, an algebraic mixing-length model with linear
Richardson-number damping---$\Lambda_c$ should be regarded as a model-specific
phase boundary rather than a universal physical constant. Diagnostics include cycle-averaged turbulent
viscosity ratio, drag coefficient, Reynolds stress, and phase-portrait attractor
classification.
\end{abstract}

\section{Introduction}

Oscillatory boundary layers driven by surface waves control the transport of
sediment in coastal and continental-shelf environments. Standard engineering
models assume fully turbulent conditions, parameterizing bed shear stress
through empirical drag coefficients and eddy-viscosity closures. However,
field measurements and laboratory experiments have demonstrated that
sufficiently high suspended-sediment concentrations can stratify the boundary
layer, suppress turbulent mixing, and trigger a transition to a laminar-like
flow regime \citep{winterwerp2001,ozdemir2010,lamb2004}. This laminarization alters effective drag, changes
sediment flux scaling, and invalidates standard closures.

Despite repeated observations, and although partial regime mappings have been
reported in DNS and LES studies \citep{ozdemir2010,cantero2009}, no systematic,
continuous parameter sweep spanning the full three-parameter space has been
performed; existing DNS and LES studies provide regime classifications at
discrete parameter points but do not map the complete $(Re, S, \Lambda)$
surface. Existing studies report thresholds in terms of local
parameters, but the regime space defined by the wave Reynolds
number $Re = U_0 \delta / \nu$, the settling number $S = w_s / U_0$, and the
stratification parameter $\Lambda = g \beta C_0 \delta / U_0^2$ (the ratio of
buoyancy forces from suspended sediment to oscillatory inertial forces) has not
been swept continuously.

We address this gap computationally. By coupling the vertically resolved
momentum equation with a sediment advection--diffusion equation and an
algebraic turbulence closure damped by the gradient Richardson number, we sweep
the $(Re, S, \Lambda)$ parameter space to identify the critical surface
$\Lambda_c(Re, S)$ separating turbulent and laminar regimes.
Although motivated by coastal sediment transport, the underlying stratification-turbulence interaction is generic, and the regime boundaries identified here may inform other stratified oscillatory shear flows, subject to the closure limitations discussed below.

\section{Background and Prior Work}

\subsection{Oscillatory boundary layers}

The classical theory of oscillatory boundary layers originates with
\citet{stokes1851}, who derived the exact laminar solution for flow driven by a
sinusoidally oscillating free stream above a flat plate. The solution reveals
that viscous effects are confined to a thin layer of thickness
$\delta_s = \sqrt{2\nu/\omega}$, the Stokes length, which serves as the
fundamental length scale for all subsequent work on oscillatory boundary layers.
This elegant analytical result remains the baseline against which turbulent
oscillatory flows are compared \citep{schlichting2017}.

The transition from laminar to turbulent flow in oscillatory boundary layers has
been investigated through a combination of experiment, linear stability
analysis, and direct computation. \citet{vonkerczek1974}
performed a linear stability analysis of the Stokes layer, establishing critical
Reynolds number estimates for the onset of instability. \citet{hino1976,hino1983} conducted pipe-flow experiments that identified a
sequence of flow regimes with increasing Reynolds number: disturbed-laminar,
intermittent-turbulent, and fully turbulent. \citet{akhavan1991a,akhavan1991b} carried out systematic experimental and
numerical investigations of transition in bounded oscillatory flows, providing
detailed measurements of the velocity field and Reynolds stresses through the
transition process. On the fully turbulent side, the landmark experiments of
\citet{jensen1989} measured velocity profiles and
turbulence statistics at high Reynolds numbers over smooth beds, establishing
benchmark data that have been widely used for model validation. \citet{sleath1987} extended experimental investigations to rough beds, while
\citet{jonsson1966} developed wave friction factor relationships,
and \citet{grantmadsen1979} formulated the widely used
wave--current boundary layer model. The
textbooks of \citet{fredsoe1992} and \citet{nielsen1992} synthesized much of this body of work into standard
references for coastal boundary layer dynamics.
The present study is restricted to hydraulically smooth beds; the additional
complications introduced by bed roughness---including form drag and modified
near-bed turbulence production---are beyond the present scope.

Numerical simulation has played an increasingly important role in elucidating
the physics of oscillatory boundary layers. \citet{spalart1989} performed early direct numerical simulations (DNS) of
oscillatory wall-bounded flows. \citet{vittori1998} used DNS
to study transition mechanisms in detail, while \citet{costamagna2003} identified coherent structures and their role in
turbulence production and transport. \citet{salon2007}
applied large-eddy simulation to the turbulent Stokes boundary layer, and
\citet{scandura2007} examined the structure of wall turbulence during
the cycle. More recently, \citet{ozdemir2014}
conducted DNS of smooth-walled Stokes layer transition, providing a detailed
characterisation of the intermittent turbulence regime. Experimentally,
\citet{carstensen2010,carstensen2012} documented
the formation and evolution of coherent structures and turbulent spots during
transition, and \citet{mier2021} contributed further observations
of transitional dynamics.

A common feature of nearly all of the studies cited above is that they consider
clear-fluid oscillatory boundary layers, in which density is uniform and the
turbulence dynamics are governed solely by shear. In many geophysical and
engineering applications, however, the boundary layer carries suspended sediment
whose concentration gradients introduce stable density stratification. The
resulting interaction between turbulence, sediment transport, and buoyancy
forces can fundamentally alter the transition process and even suppress
turbulence entirely---a phenomenon broadly termed laminarization. Despite its
practical importance, this coupled problem has received far less systematic
attention than its clear-fluid counterpart, motivating the present
investigation.

\subsection{Sediment stratification and turbulence suppression}

Suspended sediment introduces a stable density stratification that opposes
vertical turbulent mixing. The competition between stabilizing buoyancy and
destabilizing shear is quantified by the gradient Richardson number,
$Ri_g = -(g/\rho)(\partial\rho/\partial z) / (\partial u/\partial z)^2$.
\citet{miles1961} and \citet{howard1961} established that
$Ri_g > 1/4$ everywhere is a necessary condition for the stability of an
inviscid stratified shear flow, a result that underpins all subsequent
turbulence damping criteria. The foundational treatment of buoyancy effects in
turbulent flows is given by \citet{turner1973}; more recently, \citet{ivey2008} reviewed the subtle relationship between
stratification, turbulence, and irreversible mixing.

A well-known limitation of $Ri_g$-based closures arises in oscillatory flows,
where the mean shear $\partial u/\partial z$ passes through zero twice per
cycle. Near these flow-reversal phases the instantaneous $Ri_g$ becomes
singular or extremely large, even though the flow may remain dynamically
unstable due to residual turbulent kinetic energy and finite-amplitude
perturbations. This phase-dependent singularity complicates any closure that
relies on the local, instantaneous Richardson number and motivates caution in
interpreting cycle-phase-resolved results.

Observational and experimental evidence for sediment-induced stratification
effects has accumulated over several decades. \citet{trowbridge1994} documented fluid-mud dynamics on the Amazon continental
shelf, where high near-bed sediment concentrations created strong density
gradients that suppressed turbulent momentum transfer. \citet{friedrichs2000} showed that fine-sediment accumulation in coastal
seas is closely linked to boundary-layer stratification processes, and \citet{wright2006} reviewed gravity-driven transport on continental
shelves, emphasizing the role of wave-supported fluid muds. \citet{winterwerp2001,winterwerp2006} systematically examined stratification
effects across a wide range of sediment concentrations, from dilute
non-cohesive suspensions to dense cohesive muds, identifying concentration
thresholds beyond which turbulence is effectively quenched. \citet{lamb2004} measured the turbulent structure of high-density
suspensions formed under waves, providing direct evidence of damped Reynolds
stresses at elevated concentrations. \citet{dohmenjanssen1999,dohmenjanssen2002} documented grain-size-dependent
phase lags and transport rates in oscillatory sheet flow, highlighting the
role of vertical sediment distribution in controlling near-bed dynamics.

Direct numerical simulations have clarified the underlying mechanisms. \citet{cantero2009} simulated stratification effects in
sediment-laden turbulent channel flow, demonstrating progressive turbulence
suppression as the concentration---and hence the bulk Richardson
number---increased beyond a critical level. \citet{cantero2012}
extended this work toward universal criteria for turbulence suppression in
dilute turbidity currents. Most directly relevant to the present study, \citet{ozdemir2010} performed DNS of fine-particle-laden
oscillatory channel flow and showed that particle-induced density stratification
can fundamentally alter the flow regime, driving a transition from fully
turbulent to an intermittent or laminar-like state. \citet{ozdemir2010} observed relaminarization at bulk Richardson numbers of
$O(10^{-2}\text{--}10^{-1})$, which is broadly consistent with the critical
$\Lambda_c$ values found in the present study; however, direct quantitative
comparison is complicated by differences in the turbulence treatment (DNS
versus algebraic closure). In a broader context,
relaminarization phenomena across fluid mechanics were reviewed by \citet{narasimha1979}; more recent examples in non-sediment contexts
include the pulsatile pipe flow experiments of \citet{greenblattmoss2004} and the transient channel flow study of \citet{heseddighi2013}. \citet{floresriley2011}
used DNS to analyze turbulence collapse in the stably stratified surface layer.
\citet{balachandar2010} provided an extensive review of
turbulent dispersed multiphase flows, including two-way coupling effects
relevant to particle-laden boundary layers.

Despite this body of work, a systematic mapping of the transition from turbulent
to laminar-like conditions as a function of the governing dimensionless
groups---the Reynolds number $Re$, settling number $S$, and stratification
parameter $\Lambda$---has not been undertaken. Existing studies identify
thresholds in terms of local parameters for specific flow configurations, but no
unified phase diagram delineates the critical stratification
$\Lambda_c(Re, S)$ at which turbulence collapses in oscillatory boundary layers.
The present study addresses this gap through a comprehensive computational sweep
of the $(Re, S, \Lambda)$ parameter space.

\subsection{Turbulence closures for stratified boundary layers}

The simplest algebraic turbulence closure is the Prandtl mixing-length model
\citep{prandtl1925}, which prescribes the eddy viscosity as
$\nu_t = \kappa^2 z^2 |\partial u/\partial z|$, where $\kappa \approx 0.41$ is
the von K\'arm\'an constant and $z$ is the distance from the wall. \citet{vandriest1956} introduced an exponential damping factor to enforce the
correct near-wall scaling in the viscous sublayer. Despite its simplicity, this
algebraic closure remains widely used in boundary-layer studies where the
turbulence structure is well characterised by a single length scale. More
sophisticated approaches include the $k$--$\varepsilon$ model
\citep{launderspalding1974}, the $k$--$\omega$ model \citep{wilcox1988}, and the
hierarchy of second-moment closures developed by \citet{melloryamada1974,melloryamada1982}, the latter being especially prevalent
in geophysical and ocean modelling. \citet{rodi1987} reviewed turbulence
models for stratified flows, and \citet{pope2000} provides the
comprehensive modern reference for turbulent flow closure methods.

Stable density stratification suppresses vertical mixing, and this effect is
commonly parameterised through a damping function of the gradient Richardson
number $Ri_g$. One of the earliest such formulations was proposed by \citet{munkanderson1948}. A particularly transparent form is the linear
damping function $f(Ri_g) = \max(0,\, 1 - Ri_g/Ri_c)$, where the critical
value $Ri_c = 0.25$ follows from the Miles--Howard stability criterion
\citep{miles1961,howard1961}: stratification progressively suppresses turbulent
momentum transfer as $Ri_g$ increases, and the eddy viscosity vanishes entirely
when $Ri_g \geq Ri_c$. We note that $Ri_c = 0.25$ is adopted here as a model
parameter rather than a strict physical stability threshold, since the
Miles--Howard criterion is strictly inviscid and applies to linear
perturbations of a parallel shear flow. This simple closure captures the essential bifurcation
between turbulent and laminar-like states. Validation of
stratification--turbulence interaction has been provided by direct numerical
simulation and large-eddy simulation studies. \citet{armenio2002} and \citet{taylorsarkar2005}
examined stratified channel and open-channel flows, while \citet{taylorsarkar2008} extended this work to stratified Ekman layers.
\citet{brethouwer2007} analysed the scaling regimes of
strongly stratified turbulence, \citet{zontaetal2012} constructed a phase diagram for turbulence and internal
waves in stably stratified channel flow, and \citet{matervenayagamoorthy2014} proposed a unifying parameterisation framework
linking turbulent diffusivity to stratification strength. \citet{ivey2008} reviewed the broader relationship between
stratification, turbulence intensity, and mixing efficiency.

In the present study, we deliberately adopt the simplest closure that captures
the laminarization bifurcation: a mixing-length model with linear
Richardson-number suppression. Because our objective is regime
classification---identifying the critical surface $\Lambda_c(Re,\, S)$ that
separates turbulent and laminarized states---rather than quantitative prediction
of turbulent fluxes, an algebraic closure is sufficient. More elaborate models
($k$--$\varepsilon$, $k$--$\omega$, second-moment closures) may refine the
location of the critical surface and are natural targets for future work, but
they introduce additional model parameters and transport equations that are
unnecessary for the bifurcation analysis pursued here. The specific
implementation of the eddy-viscosity model and its coupling to the sediment
transport equations are detailed in the following section. It should be noted
that the location of $\Lambda_c$ is expected to be sensitive to the choice of
damping function. Alternative forms---such as exponential damping
$f(Ri_g) = \exp(-Ri_g/Ri_c)$ or quadratic suppression---would produce smoother
or more gradual turbulence attenuation and likely shift the critical surface
quantitatively, though the qualitative bifurcation between turbulent and
laminar-like states should persist.

\section{Model}

\subsection{Governing equations}

We consider a vertically resolved oscillatory boundary layer over a flat
sediment bed. The horizontal velocity $u(z,t)$ satisfies
\begin{equation}
  \partial_t u = F_0 \sin(\omega t)
  + \partial_z \left[ (\nu + \nu_t) \, \partial_z u \right],
  \label{eq:momentum}
\end{equation}
where $\nu$ is the molecular viscosity and $\nu_t$ is the turbulent eddy
viscosity. The body-force amplitude is related to the free-stream
velocity by $F_0 = U_0 \omega$, so that the unobstructed
(inviscid, unstratified) solution recovers the target oscillation
$u_\infty = U_0 \sin(\omega t)$; after non-dimensionalization with
velocity scale~$U_0$ and time scale~$1/\omega$, both $F_0$ and $\omega$
reduce to unity. The suspended-sediment concentration $C(z,t)$ evolves as
\begin{equation}
  \partial_t C + \partial_z (w_s C)
  = \partial_z \left( D_t \, \partial_z C \right),
  \label{eq:sediment}
\end{equation}
with settling velocity $w_s$ and turbulent diffusivity
$D_t = \nu_t / Sc_t$.

\subsection{Turbulence closure}

The turbulent viscosity is computed from a Prandtl mixing-length model
damped by stratification:
\begin{equation}
  \nu_t = \kappa^2 z^2 |\partial_z u| \; f(Ri_g),
  \label{eq:closure}
\end{equation}
where $\kappa = 0.41$ is the von K\'arm\'an constant and the damping
function is
\begin{equation}
  f(Ri_g) = \max\!\left(0,\; 1 - \frac{Ri_g}{Ri_c}\right),
  \qquad
  Ri_g = \frac{g \beta \, \partial_z C}{(\partial_z u)^2}.
  \label{eq:damping}
\end{equation}
Here $Ri_c \approx 0.25$ is the critical gradient Richardson number.
Where the local shear falls below a threshold
$|\partial_z u|^2 < \epsilon = 10^{-10}$, the Richardson number is set
to zero rather than evaluated, so that the damping function returns
$f = 1$ and $\nu_t$ is left undamped.  This avoids pathological
Richardson numbers in quiescent regions where the near-zero denominator
would otherwise produce spuriously large $Ri_g$ and incorrectly
suppress turbulent viscosity.  Alternative treatments of this singularity
exist, including phase-averaged $Ri_g$, capped $Ri_g$, and TKE-based
damping functions that decouple from instantaneous shear; the present
choice is the simplest option that preserves turbulence memory across
flow reversals in a minimal way.
The location of $\Lambda_c$ is sensitive to this choice at order-unity
level: for example, exponential damping $f(Ri_g) = \exp(-Ri_g/Ri_c)$
permits nonzero $\nu_t$ at all $Ri_g$, which would raise $\Lambda_c$
relative to the linear cutoff used here.

Because $\nu_t$ is proportional to $|\partial_z u|$, it vanishes
momentarily at every flow reversal regardless of the local
stratification.  This periodic reset of the turbulent viscosity
introduces an intrinsic intermittency in the mixing that may contribute
to the episodic laminarisation events observed near the transition
boundary (\S\,\ref{sec:nonmono}).

\subsection{Non-dimensionalization}

Using the Stokes layer thickness $\delta = \sqrt{2\nu/\omega}$ as the length
scale, $U_0$ as the velocity scale, $1/\omega$ as the time scale, and $C_0$
as the concentration scale, the system is governed by three dimensionless
groups:
\begin{equation}
  Re = \frac{U_0 \delta}{\nu}, \qquad
  S = \frac{w_s}{U_0}, \qquad
  \Lambda = \frac{g \beta C_0 \delta}{U_0^2}.
  \label{eq:params}
\end{equation}
Here $\beta = (\rho_s - \rho_f)/\rho_f$ is the submerged specific gravity
of the sediment (equivalently, $s - 1$ where $s = \rho_s/\rho_f$), so
that the buoyancy term $g\beta C$ represents the reduced-gravity
contribution from suspended sediment.

\subsection{Boundary conditions}

At the bed ($z = 0$): no-slip $u = 0$ and fixed reference concentration
$C = C_{\mathrm{ref}}$. At the domain top ($z = H$): free-slip
$\partial_z u = 0$ and zero sediment flux $\partial_z C = 0$.
The concentration scale $C_0$ used to define $\Lambda$ in
\eqref{eq:params} is identical to the bed boundary value
$C_{\mathrm{ref}}$; in the non-dimensional system both equal unity, so
$\Lambda$ directly measures the ratio of buoyancy to inertial forcing
at the bed.

\section{Numerical Method}

\subsection{Spatial discretization}

The vertical domain $[0, H]$ is discretized on a stretched grid using the
mapping
\begin{equation}
  z(\xi) = H \left[1 - \frac{\tanh\!\bigl(\gamma(1-\xi)\bigr)}{\tanh(\gamma)}\right],
  \qquad \xi \in [0,1],
  \label{eq:grid}
\end{equation}
where $\gamma$ controls the clustering of points near the bed. Derivatives
in physical space use second-order non-uniform finite differences applied
directly in physical $z$-coordinates, avoiding consistency errors from
chain-rule transformations through computational space. The
variable-coefficient diffusion operator
$\partial_z[\nu(z)\,\partial_z\,\cdot\,]$ is discretized using a
conservative finite-volume stencil and assembled as a tridiagonal matrix.

\subsection{Time integration}

We use an IMEX (implicit--explicit) Euler scheme. At each timestep the
turbulent viscosity $\nu_t^n$ is computed from the current fields and
frozen. The momentum equation is advanced as
\begin{equation}
  (I - \Delta t\, L_\nu)\, u^{n+1}
  = u^n + \Delta t\, F_0 \sin(\omega t^n),
  \label{eq:imex_u}
\end{equation}
where $L_\nu$ is the tridiagonal diffusion operator with effective
viscosity $\nu + \nu_t^n$. The sediment equation is advanced as
\begin{equation}
  (I - \Delta t\, L_D)\, C^{n+1}
  = C^n - \Delta t\, \partial_z(w_s\, C^n),
  \label{eq:imex_c}
\end{equation}
with $L_D$ built from $D_t = \nu_t^n / Sc_t$ and the settling flux
discretized with first-order upwind differencing. First-order upwinding
is chosen for its unconditional monotonicity, which prevents unphysical
negative concentrations and suppresses spurious oscillations near sharp
concentration gradients. The associated numerical diffusion, however,
smooths vertical concentration gradients, reducing the local gradient
Richardson number $Ri_g$ and potentially biasing the critical
stratification parameter $\Lambda_c$ toward higher values---i.e., the
model may underestimate the effectiveness of stratification in
suppressing turbulence. The net effect is a bias toward
over-stabilization: the reported $\Lambda_c$ values should therefore be
regarded as conservative upper bounds. Higher-order flux-limited schemes (e.g., TVD
methods) would reduce this diffusive bias and are a candidate for future
refinement. Both implicit systems are solved by the Thomas algorithm.

\subsection{Timestep selection}

Implicit diffusion removes the diffusive CFL restriction. The timestep is
set by the explicit settling term:
$\Delta t = \mathrm{CFL} \cdot \Delta z_{\min} / \max(w_s, U_0)$
with $\mathrm{CFL} = 0.4$.

\subsection{Cycle-averaged viscosity ratio}

The turbulent viscosity ratio used for regime classification is
$\langle\nu_t\rangle/\nu$, where $\langle\cdot\rangle$ denotes a
cycle average followed by a domain average.  The cycle average
$\bar{\nu}_t(z)$ is computed over the final oscillation period.  The
domain average is evaluated via trapezoidal integration on the
stretched grid:
\begin{equation}
  \langle\nu_t\rangle = \frac{1}{H}\int_0^H \bar{\nu}_t(z)\,\mathrm{d}z,
  \label{eq:nut_avg}
\end{equation}
which accounts for the non-uniform spacing inherent in the grid
mapping~\eqref{eq:grid}.

\section{Results}

We carried out four core phases of parameter sweeps plus three
supplementary analyses, totalling 1101 individual simulations across the
$(Re, S, \Lambda)$ parameter space.  Phase~1 performed a full
reconnaissance (270~cases at $N=256$, 20~oscillation cycles), Phase~2
verified grid convergence (24~cases at $N=32,64,128,256$), Phase~3
refined the transition brackets (377~cases at $N=256$, 20~cycles), and
Phase~4 produced publication-quality runs at the 30~cases closest to the
regime boundary ($N=256$, 40~cycles).  Phase~5 repeated the full sweep
with an exponential damping function (270~cases at $N=256$), Phase~6
computed re-entrant vertical profiles at high resolution (3~cases at
$N=256$, 40~cycles), and Phase~7 densified coverage near the re-entrant
transition edges at $S=0.005$ and $S=0.01$ (127~cases at $N=256$,
20~cycles).  Table~\ref{tab:sweep-phases} summarises each phase.

\begin{table}[htbp]
\centering
\caption{Summary of sweep phases.}
\label{tab:sweep-phases}
\begin{tabular}{clcccc}
\toprule
Phase & Purpose & $N$ & Cases & Converged \\
\midrule
1 & Reconnaissance             & 256     & 270 & 263 (97\%) \\
2 & Grid convergence           & 32--256 &  24 &  22 (92\%) \\
3 & Refined transition         & 256     & 377 & 358 (95\%) \\
4 & Production near-transition & 256     &  30 &  29 (97\%) \\
\midrule
  & Core total                 &         & 701 & 672 (96\%) \\
\midrule
5 & Exponential damping        & 256     & 270 & 267 (98\%) \\
6 & Re-entrant profiles        & 256     &   3 &   3 (100\%) \\
7 & Re-entrant expansion       & 256     & 127 & 117 (92\%) \\
\midrule
  & Grand total                &         & 1101 & 1059 (96\%) \\
\bottomrule
\end{tabular}
\end{table}

\subsection{Validation}

The solver is validated against three analytic benchmarks: (i)~convergence
of stretched-grid finite differences applied to known functions,
(ii)~implicit diffusion of a sinusoidal initial condition compared with the
analytic heat kernel, and (iii)~the Stokes oscillatory boundary layer
solution $u(z,t) = U_0 e^{-z/\delta}\sin(\omega t - z/\delta)$ run with
zero sediment coupling.

\paragraph{Turbulent baseline.}
To verify that the algebraic closure produces physically reasonable
turbulence intensities, we compare the predicted wave friction factor
$f_w = (\pi/2)\, c_f$ against the experimental measurements of
\citet{jensen1989} for clear-fluid ($\Lambda = 0$) oscillatory boundary
layers.  At $Re_\delta = 803$ ($Re_a \approx 3.2 \times 10^5$ in the
Jensen convention), the model yields $f_w \approx 0.091$, within 20\% of
the measured value $f_w = 0.114$.  At $Re_\delta = 394$ ($Re_a \approx
7.8 \times 10^4$), near the laminar--turbulent transition, the model
underestimates $f_w$ by roughly 46\% ($f_w \approx 0.121$ vs.\
$0.226$).  This discrepancy is expected: the algebraic mixing-length
closure lacks turbulent kinetic energy transport, which is particularly
important in the transitional regime where turbulence is intermittent
and phase-dependent.  For regime classification purposes---distinguishing
turbulent from laminar-like states---the order-of-magnitude agreement
is adequate, and the model correctly identifies both Reynolds numbers
as turbulent ($\langle\nu_t\rangle/\nu > 10$).

\subsection{Grid convergence}

Six representative cases spanning the laminar, turbulent, and
near-transition regimes were run at $N = 32, 64, 128,$ and $256$
(Fig.~\ref{fig:grid-convergence}).  Clearly laminar ($Re=100$) and
clearly turbulent ($Re=500$, $\Lambda=0$) cases are grid-independent:
$\langle\nu_t\rangle/\nu$ varies by less than 10\% across all
resolutions.  The threshold $\langle\nu_t\rangle/\nu = 10$ represents
approximately one order of magnitude of turbulent enhancement over
molecular viscosity, providing a physically meaningful demarcation
between laminar and turbulent states.  In the context of the momentum
equation, this threshold corresponds approximately to the onset of
order-unity Reynolds stress relative to viscous stress---i.e., the point
at which turbulent momentum transfer begins to dominate---rather than a
strict binary between turbulent and non-turbulent states.  The regime
classification is robust to moderate variations in this threshold: using
values of 8 or 12
does not alter the classification for clearly laminar or clearly
turbulent cases, although near-transition cases remain sensitive by
definition.  Phase-localized turbulence (i.e., turbulence confined to
specific phases of the oscillation cycle) was examined qualitatively in
the near-transition cases and did not alter the cycle-averaged regime
classification.  Near-transition cases, however, are sensitive to
resolution.  One of six cases changes its regime classification
between $N=32$ and $N=64$: $Re=500$, $S=0.01$, $\Lambda=0.1$
is classified as laminar at $N=32$ ($\langle\nu_t\rangle/\nu = 7.7$)
but turbulent at $N=64$ (17.3), $N=128$ (17.7), and $N=256$ (17.9).

At high Reynolds number, the turbulent viscosity ratio grows with
resolution: for $Re=1000$, $S=0.1$, $\Lambda=5.0$, the ratio increases
from 30.1 ($N=32$) to 34.5 ($N=256$), indicating that coarse grids
under-resolve turbulent transport.  Based on these results, $N=128$ is
the minimum reliable resolution for transition studies, and $N=256$ is
used for all production runs (Phases~1, 3--7).

\figGridConvergence{}

\subsection{Regime diagram}

Figure~\ref{fig:regime-diagram} presents the regime classification
across the full $(Re, S, \Lambda)$ parameter space, with one panel per
settling number $S$.  Each point represents a converged simulation from
Phases~1, 3, and~7 (666~unique converged cases at $N=256$), classified
as turbulent (red squares, $\langle\nu_t\rangle/\nu \geq 10$) or
laminar (blue circles).  Throughout this paper, ``laminar'' in the
context of regime classification denotes the laminar-like state
identified by $\langle\nu_t\rangle/\nu < 10$, which retains nonzero but
weak turbulent viscosity; it is not synonymous with the analytically
laminar Stokes solution.

\figRegimeDiagram{}

The regime map reveals three key features:

\begin{enumerate}
\item \textbf{Base transition at $Re \approx 100$--$200$.}  Below
  $Re = 200$, all cases are laminar regardless of stratification, with
  $\langle\nu_t\rangle/\nu$ between 2.2 and 8.6.  At $Re = 200$, all
  cases are weakly turbulent ($\langle\nu_t\rangle/\nu \approx 12.7$).
  Above $Re = 300$, most cases are turbulent unless sufficient
  stratification is applied.

\item \textbf{Settling number controls stratification effectiveness.}
  For slow settling ($S \leq 0.01$), sediment distributes vertically and
  stratification damps turbulence at moderate $\Lambda$.  For fast settling
  ($S = 0.5$), sediment remains confined near the bed, leaving
  turbulence undamped: all $Re = 300$ cases at $S = 0.5$ are turbulent
  ($\langle\nu_t\rangle/\nu \approx 12.1$--$12.2$ across all $\Lambda$).

\item \textbf{Non-monotonic regime boundaries.}  At $S = 0.005$ and
  $S = 0.01$, the regime classification oscillates between laminar and
  turbulent as $\Lambda$ increases for $Re \geq 300$.  For example,
  at $Re = 1000$, $S = 0.005$, the flow is turbulent for
  $\Lambda \leq 0.1$, laminar for $0.25 \leq \Lambda \leq 2.0$, and
  turbulent again at $\Lambda \geq 4.6$.  This feature persists across
  grid resolutions and integration lengths
  (see \S\,\ref{sec:nonmono}).  We note, however, that the
  re-entrant behaviour may also be amplified by the linear $Ri_g$
  damping function $f(Ri_g) = \max(0,\, 1 - Ri_g/Ri_c)$, whose sharp
  cutoff at $Ri_g = Ri_c$ can intensify feedback loops between $\nu_t$
  and $C$; a smoother closure could modify the width and sharpness of
  the re-entrant region.
\end{enumerate}

\subsection{Viscosity ratio profiles}

Figure~\ref{fig:viscosity-slices} plots the cycle-averaged viscosity
ratio against $\Lambda$ for six $(Re, S)$ slices, combining Phases~1, 3, and~7 data at $N = 256$.  The dashed line marks the classification
threshold at $\langle\nu_t\rangle/\nu = 10$.

\figViscositySlices{}

For low settling numbers ($S = 0.005$ at $Re = 300$ and $S = 0.01$ at
$Re = 500$), the viscosity ratio shows non-monotonic behaviour:
$\langle\nu_t\rangle/\nu$ drops below 10 at intermediate $\Lambda$,
then recovers at higher $\Lambda$.  At higher settling numbers ($S = 0.05$,
$S = 0.1$, $S = 0.5$), all cases remain turbulent across the full
$\Lambda$ range, as fast settling confines sediment near the bed and
prevents the outer-layer stratification that drives laminarization.

\subsection{Critical stratification}

The corrected settling flux produces a transition topology that differs
qualitatively from a simple critical $\Lambda_c$.  Rather than a sharp,
monotonic boundary between turbulent and laminar states, most $(Re, S)$
combinations exhibit non-monotonic behaviour: the flow transitions from
turbulent to laminar at a first onset $\Lambda_\ell$ and, at higher
$\Lambda$, recovers to a re-entrant turbulent state at $\Lambda_r$
(Fig.~\ref{fig:critical-lambda}).  Table~\ref{tab:lambda-c} summarises
this topology.

\begin{table}[htbp]
\centering
\caption{Transition topology from Phases~1, 3, and~7 ($N=256$) for low
settling numbers.  $\Lambda_\ell$: first laminar onset; $\Lambda_r$:
re-entrant turbulent recovery.  A dash indicates no laminar regime was
observed.  Phase~7 densified coverage near transition edges, resolving
the Re=750 case and confirming two-band structure at Re=300.}
\label{tab:lambda-c}
\begin{tabular}{cccccl}
\toprule
$Re$ & $S$ & $\Lambda_\ell$ & $\Lambda_r$ & $N_{\text{lam}}$ & Notes \\
\midrule
 300 & 0.005 & $\sim 0.029$ & $\sim 0.061$ & 15 & Two bands$^\dagger$ \\
 500 & 0.005 & $\sim 0.30$ & $\sim 1.26$  & 9 & \\
 750 & 0.005 & $\sim 0.30$ & $\sim 1.78$  & 4 & Resolved by Phase~7 \\
1000 & 0.005 & $\sim 0.28$ & $\sim 2.55$ & 4 & Wide laminar band \\
 300 & 0.01  & $\sim 0.061$ & $\sim 0.14$ & 13 & \\
 500 & 0.01  & ---         & ---         & 0 & No laminar at $N=256$ \\
 750 & 0.01  & ---         & ---         & 0 & No laminar at $N=256$ \\
1000 & 0.01  & ---         & ---         & 0 & No laminar at $N=256$ \\
\bottomrule
\multicolumn{6}{l}{\footnotesize $^\dagger$ Two disjoint $\Lambda$ intervals produce laminar states; second band: $\Lambda_\ell \sim 0.50$, $\Lambda_r \sim 0.68$.}
\end{tabular}
\end{table}

\figCriticalLambda{}

The first laminar onset is $O(0.1\text{--}2.0)$, substantially higher
than the pre-correction estimates of $O(0.01\text{--}0.1)$.  The
corrected settling flux reduces the effective stratification at a given
$\Lambda$, requiring stronger sediment loading to trigger
laminarization.  A single critical $\Lambda_c$ is ill-posed at most
$(Re, S)$ because the transition is non-monotonic: the flow laminarizes
at intermediate $\Lambda$ but recovers turbulence at higher $\Lambda$
where the concentration profile adjusts to reduce $Ri_g$
(\S\,\ref{sec:nonmono}).  The non-monotonicity is pervasive, appearing
at $S = 0.005$ and $S = 0.01$ across all Reynolds numbers tested, not
merely at $S \geq 0.1$ as in the pre-correction results.

\subsection{Production diagnostics}
\label{sec:production}

Table~\ref{tab:phase4} presents the full diagnostic output from the
30~Phase~4 production runs ($N=256$, 40~cycles), including the drag
coefficient $c_f$ and cycle-averaged kinetic energy $\langle KE\rangle$.
29 of 30 cases converged, a marked improvement over Phase~3 (67\%
convergence), attributable to the higher resolution and longer
integration.  All 30~cases are classified as turbulent; the $Re=200$ cases, previously
classified as laminar under the old viscosity ratio averaging, now yield
$\langle\nu_t\rangle/\nu \approx 12.7$ and are weakly turbulent.

\begin{table}[htbp]
\centering
\caption{Phase~4 production diagnostics ($N=256$, 40~cycles).
$c_f = \tau_{\mathrm{bed}} / (\frac{1}{2} U_0^2)$;
$\langle KE\rangle = \langle\int\frac{1}{2}u^2\,\mathrm{d}z\rangle$.
L: laminar ($\langle\nu_t\rangle/\nu < 10$);
T: turbulent ($\langle\nu_t\rangle/\nu \geq 10$).}
\label{tab:phase4}
\small
\begin{tabular}{rrlcrrr}
\toprule
$Re$ & $S$ & $\Lambda$ & Regime & $\langle\nu_t\rangle/\nu$ & $c_f$ & $\langle KE\rangle$ \\
\midrule
 300 & 0.005 & 0.25 & T & 13.7 & 0.0861 & 2.802 \\
 300 & 0.005 & 0.75 & T & 15.9 & 0.0862 & 2.677 \\
 300 & 0.005 & 1.25 & L &  9.3 & 0.0872 & 3.276 \\
 300 & 0.005 & 2.50 & T & 18.4 & 0.0864 & 2.831 \\
\midrule
 500 & 0.005 & 0.10 & T & 12.7 & 0.0696 & 3.246 \\
 500 & 0.005 & 0.50 & L &  8.8 & 0.0693 & 3.107 \\
 500 & 0.005 & 1.25 & T & 19.8 & 0.0693 & 2.538 \\
 500 & 0.005 & 2.00 & T & 25.1 & 0.0695 & 2.739 \\
 500 & 0.005 & 3.00 & T & 28.4 & 0.0697 & 2.868 \\
 500 & 0.010 & 0.25 & T & 12.4 & 0.0697 & 3.250 \\
 500 & 0.010 & 0.75 & T & 10.9 & 0.0697 & 3.201 \\
 500 & 0.010 & 1.25 & T & 22.4 & 0.0695 & 2.712 \\
 500 & 0.010 & 2.50 & T & 27.8 & 0.0696 & 2.821 \\
\midrule
 750 & 0.005 & 0.10 & T & 24.5 & 0.0583 & 2.845 \\
 750 & 0.005 & 0.50 & L &  6.6 & 0.0581 & 2.901 \\
 750 & 0.005 & 1.50 & L &  7.3 & 0.0587 & 3.047 \\
 750 & 0.005 & 3.00 & T & 15.5 & 0.0595 & 3.240 \\
 750 & 0.010 & 0.10 & T & 31.0 & 0.0584 & 2.907 \\
 750 & 0.010 & 0.50 & T & 18.5 & 0.0583 & 2.824 \\
 750 & 0.010 & 1.50 & T & 11.8 & 0.0590 & 3.175 \\
 750 & 0.010 & 3.50 & T & 18.3 & 0.0598 & 3.333 \\
\midrule
1000 & 0.005 & 0.10 & T & 19.1 & 0.0522 & 3.232 \\
1000 & 0.005 & 0.50 & L &  4.9 & 0.0517 & 2.904 \\
1000 & 0.005 & 1.50 & L &  3.7 & 0.0522 & 3.016 \\
1000 & 0.005 & 3.00 & T & 13.5 & 0.0529 & 3.161 \\
1000 & 0.005 & 5.00 & T & 20.5 & 0.0533 & 3.292 \\
1000 & 0.010 & 0.10 & T & 22.1 & 0.0525 & 3.325 \\
1000 & 0.010 & 0.50 & T & 13.7 & 0.0520 & 3.150 \\
1000 & 0.010 & 1.50 & T & 10.7 & 0.0523 & 3.107 \\
1000 & 0.010 & 3.00 & T & 16.8 & 0.0531 & 3.239 \\
\bottomrule
\end{tabular}
\end{table}

Figure~\ref{fig:production-diagnostics} shows the drag coefficient and
kinetic energy as functions of the viscosity ratio for all Phase~4
cases.

\figProductionDiagnostics{}

The drag coefficient decreases systematically with $Re$ for the
turbulent cases: $c_f \approx 0.086$ at $Re = 300$,
$c_f \approx 0.070$ at $Re = 500$, $c_f \approx 0.058$--$0.059$ at
$Re = 750$, and $c_f \approx 0.052$--$0.054$ at $Re = 1000$,
consistent with the expected scaling $c_f \sim Re^{-1/2}$ for
oscillatory boundary layers.  The weakly turbulent $Re = 200$ cases have
$c_f \approx 0.103$, higher than all other turbulent cases, consistent with
the thicker boundary layer at low $Re$.
Within each $Re$, the drag coefficient is nearly insensitive to $S$ and
$\Lambda$, confirming that near-bed shear is dominated by the oscillatory
forcing rather than the turbulent enhancement at these Reynolds numbers.
Consequently, $c_f$ is a poor discriminator of the laminarization
transition; turbulent viscosity ratio or Reynolds stress diagnostics are
more reliable indicators.

The kinetic energy spans $\langle KE\rangle = 2.07$--$3.03$ across
all cases.  The weakly turbulent $Re = 200$ cases cluster at
$\langle KE\rangle \approx 2.66$, within the overall range.  The
lowest values occur at high $Re$ and near-transition $\Lambda$ (e.g.,
$Re = 1000$, $S = 0.01$, $\Lambda = 0.25$: $\langle KE\rangle = 2.07$)
where stratification partially suppresses the velocity profile even
though the flow remains turbulent.  Cases far from the transition
boundary cluster near $\langle KE\rangle \approx 2.8$--$3.0$.

\subsection{Phase portraits}
\label{sec:phase-portraits}

Figure~\ref{fig:phase-portraits} shows the cycle-averaged phase
portraits $(u_{\mathrm{bed}},\, \tau_{\mathrm{bed}})$ for four
representative Phase~4 cases spanning the weakly and strongly turbulent regimes.

\figPhasePortraits{}

The Phase~4 set spans weakly turbulent ($Re=200$) to strongly turbulent
($Re \geq 300$) cases, and the phase portraits illustrate the spectrum
of turbulent intensities.  The weakly turbulent
$Re = 200$ cases ($\langle\nu_t\rangle/\nu \approx 12.7$) produce
tight elliptical orbits with low $\tau_{\mathrm{bed}}$ amplitude,
while the strongly turbulent case ($Re=1000$, $S=0.01$,
$\Lambda=5.0$, $\langle\nu_t\rangle/\nu = 51.4$) shows a broader
loop with substantially higher bed shear stress.  This continuous
variation in attractor geometry is consistent with the algebraic
closure model, which produces a smooth $\nu_t$ field rather than a
discrete switch between regimes.

\subsection{Non-monotonic regime behaviour}
\label{sec:nonmono}

A pervasive feature of the corrected results is non-monotonic regime
behaviour at low settling numbers ($S = 0.005$ and $S = 0.01$) across
all Reynolds numbers above the base transition.  The most striking
example is $Re = 1000$, $S = 0.005$, where the combined
Phase~1/3/7 data show: turbulent for $\Lambda \leq 0.27$
($\langle\nu_t\rangle/\nu = 10.3$--$51.6$), laminar for
$0.28 \leq \Lambda \leq 2.5$ ($\langle\nu_t\rangle/\nu = 8.9$--$9.9$),
and turbulent again at $\Lambda \geq 2.6$
($\langle\nu_t\rangle/\nu = 10.4$--$51.5$).  This wide laminar band
spanning nearly a decade in $\Lambda$ is bounded on both sides by
turbulent states.

This non-monotonicity survives across all resolutions from $N = 64$
through $N = 256$, and persists from 20 to 40~oscillation cycles.  We
interpret this as a consequence of competing effects: at intermediate
$\Lambda$, stratification damps the mixing length and $Ri_g$ exceeds
$Ri_c$ across much of the boundary layer; at high $\Lambda$, the very
strong sediment loading produces a sharp concentration gradient confined
near the bed, which reduces $Ri_g$ in the outer layer and allows
turbulence to re-establish.

Figure~\ref{fig:reentrant_profiles} shows the cycle-averaged vertical
profiles of concentration $C(z)$, gradient Richardson number $Ri_g(z)$,
and eddy viscosity $\nu_t(z)$ for three values of $\Lambda$ at
$(Re, S) = (1000, 0.005)$, computed on a fine grid ($N = 256$,
40~cycles).  At $\Lambda = 0.1$ (turbulent,
$\langle\nu_t\rangle/\nu = 19.9$), the stratification is insufficient
to push $Ri_g$ above $Ri_c$ over most of the boundary layer, and $\nu_t$
remains elevated.  At $\Lambda = 2.0$ (laminar,
$\langle\nu_t\rangle/\nu = 7.8$), the stronger sediment loading
produces a concentration profile whose gradient raises $Ri_g$ above
$Ri_c$, collapsing $\nu_t$.  At $\Lambda = 5.0$ (re-entrant turbulent,
$\langle\nu_t\rangle/\nu = 51.4$), the still-stronger sediment loading
concentrates sediment so sharply near the bed that the outer-layer
concentration gradient is actually reduced; $Ri_g$ falls below $Ri_c$ in
the outer layer, and turbulence recovers with vigour.

\figReentrantProfiles{}

These profiles confirm the proposed mechanism: the non-monotonicity arises
from the competition between stratification strength (which increases
monotonically with $\Lambda$) and the confinement of sediment near the bed
(which at high $\Lambda$ reduces the outer-layer concentration gradient
and hence $Ri_g$).  The sensitivity analysis in
\S\,\ref{sec:damping_sensitivity} further confirms that this re-entrant
behaviour is robust to the choice of damping function.

\section{Discussion and Scaling}

Although motivated by sediment-laden coastal boundary layers, the
stratification--shear interaction explored here is generic: the regime
boundaries and non-monotonic topology are properties of
Richardson-number-damped oscillatory shear flows and may apply to other
stratified oscillatory systems (e.g., thermally stratified tidal boundary
layers or particle-laden pulsatile flows).

\subsection{Critical stratification scaling}

The critical stratification parameter $\Lambda_c$ is extracted as the
boundary between turbulent and laminar regimes in the phase diagram.  We
seek a reduced scaling law of the form
\begin{equation}
  \Lambda_c = f(Re,\, S),
  \label{eq:scaling}
\end{equation}
fitted to the computed regime boundary.  Table~\ref{tab:lambda-c} and
Fig.~\ref{fig:critical-lambda} show $\Lambda_c(Re)$ for the two
lowest settling numbers ($S = 0.005$ and $S = 0.01$), where the
transition is sharp and well-defined.

The first laminar onset $\Lambda_\ell$ is $O(0.1\text{--}2.0)$,
substantially higher than pre-correction estimates, and varies with
both $Re$ and $S$ (Table~\ref{tab:lambda-c}).  At fixed $S$, the onset
generally shifts to lower $\Lambda$ with increasing $Re$, but this
trend is not monotonic.  The settling number controls how effectively
sediment distributes vertically to establish the stratification that
damps the mixing length.

This saturation is reminiscent of the flux Richardson number ceiling observed
in stratified shear layers, where turbulence self-limits at
$Ri_f \approx 0.2$--$0.25$ regardless of the imposed shear
\citep{osborn1980}. In the present model the analogous constraint is the
critical gradient Richardson number $Ri_c = 0.25$ of the linear damping
function; once the flow is sufficiently energetic to maintain turbulence, the
onset of laminarisation depends on whether the sediment-induced $Ri_g$ can
exceed $Ri_c$ locally, which is governed by $\Lambda$ and $S$ rather than
$Re$.

Power-law fits of the form $\Lambda_\ell \sim Re^a S^b$ do not produce
a convincing collapse.  Moreover, the non-monotonic transition topology
renders a single scaling law ill-posed: the first laminar onset
$\Lambda_\ell$ does not uniquely characterise the regime boundary because
turbulence recovers at a re-entrant value $\Lambda_r > \Lambda_\ell$ at
most $(Re, S)$.  Any reduced description must account for both the onset
and recovery boundaries, which likely require separate parameterisations.
DNS or LES at selected $(Re, S, \Lambda)$ points near the predicted
phase boundary would provide the most direct test of the
algebraic-closure predictions and help distinguish physical from
closure-induced features of the regime map.

\subsection{Drag coefficient behaviour}

The production diagnostics (Table~\ref{tab:phase4},
Fig.~\ref{fig:production-diagnostics}a) reveal that among the turbulent
cases, $c_f$ at $Re = 300$ is nearly uniform ($c_f \approx 0.086$),
consistent with the oscillatory boundary layer structure at moderate
$Re$, where the bed shear stress is dominated by the
pressure-gradient-driven oscillation rather than turbulent Reynolds
stresses.  The weakly turbulent $Re = 200$ cases have $c_f \approx 0.103$,
higher than all other turbulent cases, reflecting the thicker boundary
layer at low $Re$.

At higher $Re$, $c_f$ decreases systematically: $c_f \approx 0.070$ at
$Re = 500$, $c_f \approx 0.058$--$0.059$ at $Re = 750$, and
$c_f \approx 0.052$--$0.054$ at $Re = 1000$.  This reduction follows the
expected scaling $c_f \sim Re^{-1/2}$ for oscillatory boundary layers,
where the viscous sublayer thins relative to the Stokes layer thickness
$\delta_s = \sqrt{2\nu/\omega}$.

\subsection{Kinetic energy and stratification}

The cycle-averaged kinetic energy $\langle KE \rangle$
(Fig.~\ref{fig:production-diagnostics}b) spans
$\langle KE\rangle = 2.07$--$3.03$ across all Phase~4 cases, with
cases far from the transition boundary clustering near
$\langle KE\rangle \approx 2.8$--$3.0$.  The lowest kinetic energies
correspond to high-$Re$ cases near the transition ($Re = 1000$,
$S = 0.01$, $\Lambda = 0.25$: $\langle KE\rangle = 2.07$), where
stratification partially suppresses the velocity profile even though
the flow remains turbulent.

\subsection{Non-monotonic transitions and re-entrant boundaries}

The non-monotonic regime behaviour identified at $S = 0.005$
across all $Re \geq 300$, and at $S = 0.01$ for $Re = 300$,
(Fig.~\ref{fig:viscosity-slices}, \S\,\ref{sec:nonmono})
is the most unexpected finding of this study.  Rather than a single
critical $\Lambda_c$ separating turbulent and laminar regimes, the
phase boundary is re-entrant: the flow laminarizes at intermediate
$\Lambda$ but recovers turbulence at high $\Lambda$.

We attribute this to the competition between two mechanisms:
\begin{enumerate}
\item \textbf{Direct stratification damping.}  Increasing $\Lambda$
  strengthens the stable density gradient, raising $Ri_g$ and damping
  the mixing length via $f(Ri_g) = \max(0,\, 1 - Ri_g/Ri_c)$.

\item \textbf{Concentration profile adjustment.}  At moderate
  $\Lambda$, the sediment concentration profile adjusts:  enhanced
  turbulent diffusion redistributes sediment, reducing the concentration
  gradient $\partial_z C$ and hence $Ri_g$ in the outer layer.  The
  resulting reduction in stratification allows turbulence to re-establish.
\end{enumerate}

At high $S$ (fast settling), sediment remains confined near the bed
regardless of $\Lambda$, so the stratification never extends into the
outer layer.  This explains why $S = 0.5$ cases at $Re = 300$ remain
uniformly turbulent ($\langle\nu_t\rangle/\nu \approx 12.1$--$12.2$)
across all $\Lambda$: fast settling prevents the sediment
redistribution required for either laminarization or re-entrance.

\subsection{Phase portrait classification}

The phase portraits (Fig.~\ref{fig:phase-portraits}) provide a
complementary view of the regime structure.  The $(u_{\mathrm{bed}},
\tau_{\mathrm{bed}})$ trajectory traces an elliptical orbit whose
shape reflects the balance between oscillatory forcing and turbulent
dissipation.  The Phase~4 set spans weakly turbulent ($Re=200$,
$\langle\nu_t\rangle/\nu \approx 12.7$) to strongly turbulent
($Re \geq 300$) cases, and the portraits illustrate the spectrum
of turbulent intensities: near-threshold turbulent cases
($\langle\nu_t\rangle/\nu \approx 16.5$) produce tighter orbits
with lower $\tau_{\mathrm{bed}}$ amplitude, while strongly turbulent
cases ($\langle\nu_t\rangle/\nu \approx 51.4$) produce broader loops.
This continuous variation is consistent with the algebraic closure model,
which produces a smooth $\nu_t$ field.  A more physically complete model
(e.g., $k$-$\omega$) might produce sharper hysteresis in the phase
portrait, which would be an informative comparison for future work.

\subsection{Sensitivity to damping function}
\label{sec:damping_sensitivity}

To assess whether the regime topology depends on the choice of stratification
damping function, we repeat the full parameter sweep replacing the linear
closure $f(Ri_g) = \max(0,\, 1 - Ri_g/Ri_c)$ with a continuous exponential
form $f(Ri_g) = \exp(-Ri_g/Ri_c)$, which permits nonzero eddy viscosity at
all~$Ri_g$.  Figure~\ref{fig:damping_comparison} compares the regime
classifications at $S = 0.005$ and $S = 0.01$.

\figDampingComparison{}

The qualitative phase topology is preserved under exponential damping,
with the first laminar onset $\Lambda_\ell$ shifting to substantially
higher values.  At $S = 0.005$, for example, $\Lambda_\ell \approx 0.25$
with linear damping at $Re = 1000$ shifts to $\Lambda_\ell \approx 1.0$
with exponential damping; at lower $Re$, the exponential closure delays
the transition to $\Lambda \geq 2$--$5$.  The shift reflects the softer
cutoff of the exponential function, which sustains nonzero $\nu_t$ at
supercritical $Ri_g$ and thereby delays the collapse into the laminar
state.

The non-monotonic topology is also modified: the re-entrant behaviour
observed under linear damping at low $S$ is suppressed under exponential
damping, where the sustained $\nu_t$ at high $Ri_g$ prevents the sharp
feedback that drives the re-entrance.  At $Re = 300$, $S = 0.1$, all
cases remain turbulent under exponential damping.  This suggests that
while the laminarization transition is a robust feature of both closures,
the re-entrant recovery is sensitive to the sharpness of the damping
function and may be amplified by the hard cutoff in the linear closure.
While the concentration-profile adjustment mechanism is physically
plausible, the existence and width of the re-entrant turbulent band
depend sensitively on the sharpness of Richardson-number damping.  The
present results therefore demonstrate a \emph{closure-robust}
laminarization transition---the onset of turbulence suppression at
intermediate $\Lambda$ is preserved across both closures---but only a
\emph{closure-contingent} re-entrant recovery, whose quantitative
boundaries require validation against DNS or higher-fidelity closures.

\subsection{Limitations and outlook}

Several limitations of the present study should be noted.  First, the
one-dimensional model with algebraic closure cannot capture
three-dimensional turbulent structures or secondary instabilities that
may be important near the transition.  Second, the classification
threshold $\langle\nu_t\rangle/\nu = 10$ is a pragmatic choice; a more
physical criterion based on turbulent kinetic energy budgets would
strengthen the regime identification.  Third, the non-monotonic
behaviour at low $S$ warrants investigation with longer integrations or
ensemble runs to characterise the intermittent regime statistically, and
the sensitivity of re-entrant behaviour to the damping function
(\S\,\ref{sec:damping_sensitivity}) suggests that the closure choice
materially affects the predicted topology.

Future work should address: (i)~scaling collapse of the transition data
to test whether a combined parameter $\Lambda_c S^\alpha Re^\beta$
unifies the regime boundary, (ii)~phase portrait analysis for
hysteresis detection through forward and backward $\Lambda$ ramps,
and (iii)~comparison with DNS results at selected $(Re, S, \Lambda)$
points to assess the fidelity of the algebraic closure near the
bifurcation.


\section*{Data Availability}
All simulation code and data are publicly available at
\url{https://github.com/slink/laminarization-transition-study}.

\bibliographystyle{plainnat}
\bibliography{references}

\end{document}
