\section{Discussion and Scaling}

Although motivated by sediment-laden coastal boundary layers, the
stratification--shear interaction explored here is generic: the regime
boundaries and non-monotonic topology are properties of
Richardson-number-damped oscillatory shear flows and may apply to other
stratified oscillatory systems (e.g., thermally stratified tidal boundary
layers or particle-laden pulsatile flows).

\subsection{Critical stratification scaling}

The critical stratification parameter $\Lambda_c$ is extracted as the
boundary between turbulent and laminar regimes in the phase diagram.  We
seek a reduced scaling law of the form
\begin{equation}
  \Lambda_c = f(Re,\, S),
  \label{eq:scaling}
\end{equation}
fitted to the computed regime boundary.  Table~\ref{tab:lambda-c} and
Fig.~\ref{fig:critical-lambda} show $\Lambda_c(Re)$ for the two
lowest settling numbers ($S = 0.005$ and $S = 0.01$), where the
transition is sharp and well-defined.

The first laminar onset $\Lambda_\ell$ is $O(0.1\text{--}2.0)$,
substantially higher than pre-correction estimates, and varies with
both $Re$ and $S$ (Table~\ref{tab:lambda-c}).  At fixed $S$, the onset
generally shifts to lower $\Lambda$ with increasing $Re$, but this
trend is not monotonic.  The settling number controls how effectively
sediment distributes vertically to establish the stratification that
damps the mixing length.

This saturation is reminiscent of the flux Richardson number ceiling observed
in stratified shear layers, where turbulence self-limits at
$Ri_f \approx 0.2$--$0.25$ regardless of the imposed shear
\citep{osborn1980}. In the present model the analogous constraint is the
critical gradient Richardson number $Ri_c = 0.25$ of the linear damping
function; once the flow is sufficiently energetic to maintain turbulence, the
onset of laminarisation depends on whether the sediment-induced $Ri_g$ can
exceed $Ri_c$ locally, which is governed by $\Lambda$ and $S$ rather than
$Re$.

Power-law fits of the form $\Lambda_\ell \sim Re^a S^b$ do not produce
a convincing collapse.  Moreover, the non-monotonic transition topology
renders a single scaling law ill-posed: the first laminar onset
$\Lambda_\ell$ does not uniquely characterise the regime boundary because
turbulence recovers at a re-entrant value $\Lambda_r > \Lambda_\ell$ at
most $(Re, S)$.  Any reduced description must account for both the onset
and recovery boundaries, which likely require separate parameterisations.
DNS or LES at selected $(Re, S, \Lambda)$ points near the predicted
phase boundary would provide the most direct test of the
algebraic-closure predictions and help distinguish physical from
closure-induced features of the regime map.

\subsection{Drag coefficient behaviour}

The production diagnostics (Table~\ref{tab:phase4},
Fig.~\ref{fig:production-diagnostics}a) reveal that among the turbulent
cases, $c_f$ at $Re = 300$ is nearly uniform ($c_f \approx 0.086$),
consistent with the oscillatory boundary layer structure at moderate
$Re$, where the bed shear stress is dominated by the
pressure-gradient-driven oscillation rather than turbulent Reynolds
stresses.  The weakly turbulent $Re = 200$ cases have $c_f \approx 0.103$,
higher than all other turbulent cases, reflecting the thicker boundary
layer at low $Re$.

At higher $Re$, $c_f$ decreases systematically: $c_f \approx 0.070$ at
$Re = 500$, $c_f \approx 0.058$--$0.059$ at $Re = 750$, and
$c_f \approx 0.052$--$0.054$ at $Re = 1000$.  This reduction follows the
expected scaling $c_f \sim Re^{-1/2}$ for oscillatory boundary layers,
where the viscous sublayer thins relative to the Stokes layer thickness
$\delta_s = \sqrt{2\nu/\omega}$.

\subsection{Kinetic energy and stratification}

The cycle-averaged kinetic energy $\langle KE \rangle$
(Fig.~\ref{fig:production-diagnostics}b) spans
$\langle KE\rangle = 2.07$--$3.03$ across all Phase~4 cases, with
cases far from the transition boundary clustering near
$\langle KE\rangle \approx 2.8$--$3.0$.  The lowest kinetic energies
correspond to high-$Re$ cases near the transition ($Re = 1000$,
$S = 0.01$, $\Lambda = 0.25$: $\langle KE\rangle = 2.07$), where
stratification partially suppresses the velocity profile even though
the flow remains turbulent.

\subsection{Non-monotonic transitions and re-entrant boundaries}

The non-monotonic regime behaviour identified at $S = 0.005$
across all $Re \geq 300$, and at $S = 0.01$ for $Re = 300$,
(Fig.~\ref{fig:viscosity-slices}, \S\,\ref{sec:nonmono})
is the most unexpected finding of this study.  Rather than a single
critical $\Lambda_c$ separating turbulent and laminar regimes, the
phase boundary is re-entrant: the flow laminarizes at intermediate
$\Lambda$ but recovers turbulence at high $\Lambda$.

We attribute this to the competition between two mechanisms:
\begin{enumerate}
\item \textbf{Direct stratification damping.}  Increasing $\Lambda$
  strengthens the stable density gradient, raising $Ri_g$ and damping
  the mixing length via $f(Ri_g) = \max(0,\, 1 - Ri_g/Ri_c)$.

\item \textbf{Concentration profile adjustment.}  At moderate
  $\Lambda$, the sediment concentration profile adjusts:  enhanced
  turbulent diffusion redistributes sediment, reducing the concentration
  gradient $\partial_z C$ and hence $Ri_g$ in the outer layer.  The
  resulting reduction in stratification allows turbulence to re-establish.
\end{enumerate}

At high $S$ (fast settling), sediment remains confined near the bed
regardless of $\Lambda$, so the stratification never extends into the
outer layer.  This explains why $S = 0.5$ cases at $Re = 300$ remain
uniformly turbulent ($\langle\nu_t\rangle/\nu \approx 12.1$--$12.2$)
across all $\Lambda$: fast settling prevents the sediment
redistribution required for either laminarization or re-entrance.

\subsection{Phase portrait classification}

The phase portraits (Fig.~\ref{fig:phase-portraits}) provide a
complementary view of the regime structure.  The $(u_{\mathrm{bed}},
\tau_{\mathrm{bed}})$ trajectory traces an elliptical orbit whose
shape reflects the balance between oscillatory forcing and turbulent
dissipation.  The Phase~4 set spans weakly turbulent ($Re=200$,
$\langle\nu_t\rangle/\nu \approx 12.7$) to strongly turbulent
($Re \geq 300$) cases, and the portraits illustrate the spectrum
of turbulent intensities: near-threshold turbulent cases
($\langle\nu_t\rangle/\nu \approx 16.5$) produce tighter orbits
with lower $\tau_{\mathrm{bed}}$ amplitude, while strongly turbulent
cases ($\langle\nu_t\rangle/\nu \approx 51.4$) produce broader loops.
This continuous variation is consistent with the algebraic closure model,
which produces a smooth $\nu_t$ field.  A more physically complete model
(e.g., $k$-$\omega$) might produce sharper hysteresis in the phase
portrait, which would be an informative comparison for future work.

\subsection{Sensitivity to damping function}
\label{sec:damping_sensitivity}

To assess whether the regime topology depends on the choice of stratification
damping function, we repeat the full parameter sweep replacing the linear
closure $f(Ri_g) = \max(0,\, 1 - Ri_g/Ri_c)$ with a continuous exponential
form $f(Ri_g) = \exp(-Ri_g/Ri_c)$, which permits nonzero eddy viscosity at
all~$Ri_g$.  Figure~\ref{fig:damping_comparison} compares the regime
classifications at $S = 0.005$ and $S = 0.01$.

\figDampingComparison{}

The qualitative phase topology is preserved under exponential damping,
with the first laminar onset $\Lambda_\ell$ shifting to substantially
higher values.  At $S = 0.005$, for example, $\Lambda_\ell \approx 0.25$
with linear damping at $Re = 1000$ shifts to $\Lambda_\ell \approx 1.0$
with exponential damping; at lower $Re$, the exponential closure delays
the transition to $\Lambda \geq 2$--$5$.  The shift reflects the softer
cutoff of the exponential function, which sustains nonzero $\nu_t$ at
supercritical $Ri_g$ and thereby delays the collapse into the laminar
state.

The non-monotonic topology is also modified: the re-entrant behaviour
observed under linear damping at low $S$ is suppressed under exponential
damping, where the sustained $\nu_t$ at high $Ri_g$ prevents the sharp
feedback that drives the re-entrance.  At $Re = 300$, $S = 0.1$, all
cases remain turbulent under exponential damping.  This suggests that
while the laminarization transition is a robust feature of both closures,
the re-entrant recovery is sensitive to the sharpness of the damping
function and may be amplified by the hard cutoff in the linear closure.
While the concentration-profile adjustment mechanism is physically
plausible, the existence and width of the re-entrant turbulent band
depend sensitively on the sharpness of Richardson-number damping.  The
present results therefore demonstrate a \emph{closure-robust}
laminarization transition---the onset of turbulence suppression at
intermediate $\Lambda$ is preserved across both closures---but only a
\emph{closure-contingent} re-entrant recovery, whose quantitative
boundaries require validation against DNS or higher-fidelity closures.

\subsection{Limitations and outlook}

Several limitations of the present study should be noted.  First, the
one-dimensional model with algebraic closure cannot capture
three-dimensional turbulent structures or secondary instabilities that
may be important near the transition.  Second, the classification
threshold $\langle\nu_t\rangle/\nu = 10$ is a pragmatic choice; a more
physical criterion based on turbulent kinetic energy budgets would
strengthen the regime identification.  Third, the non-monotonic
behaviour at low $S$ warrants investigation with longer integrations or
ensemble runs to characterise the intermittent regime statistically, and
the sensitivity of re-entrant behaviour to the damping function
(\S\,\ref{sec:damping_sensitivity}) suggests that the closure choice
materially affects the predicted topology.

Future work should address: (i)~scaling collapse of the transition data
to test whether a combined parameter $\Lambda_c S^\alpha Re^\beta$
unifies the regime boundary, (ii)~phase portrait analysis for
hysteresis detection through forward and backward $\Lambda$ ramps,
and (iii)~comparison with DNS results at selected $(Re, S, \Lambda)$
points to assess the fidelity of the algebraic closure near the
bifurcation.
