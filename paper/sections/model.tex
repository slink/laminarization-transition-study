\section{Model}

\subsection{Governing equations}

We consider a vertically resolved oscillatory boundary layer over a flat
sediment bed. The horizontal velocity $u(z,t)$ satisfies
\begin{equation}
  \partial_t u = F_0 \sin(\omega t)
  + \partial_z \left[ (\nu + \nu_t) \, \partial_z u \right],
  \label{eq:momentum}
\end{equation}
where $\nu$ is the molecular viscosity and $\nu_t$ is the turbulent eddy
viscosity. The body-force amplitude is related to the free-stream
velocity by $F_0 = U_0 \omega$, so that the unobstructed
(inviscid, unstratified) solution recovers the target oscillation
$u_\infty = U_0 \sin(\omega t)$; after non-dimensionalization with
velocity scale~$U_0$ and time scale~$1/\omega$, both $F_0$ and $\omega$
reduce to unity. The suspended-sediment concentration $C(z,t)$ evolves as
\begin{equation}
  \partial_t C + \partial_z (w_s C)
  = \partial_z \left( D_t \, \partial_z C \right),
  \label{eq:sediment}
\end{equation}
with settling velocity $w_s$ and turbulent diffusivity
$D_t = \nu_t / Sc_t$.

\subsection{Turbulence closure}

The turbulent viscosity is computed from a Prandtl mixing-length model
damped by stratification:
\begin{equation}
  \nu_t = \kappa^2 z^2 |\partial_z u| \; f(Ri_g),
  \label{eq:closure}
\end{equation}
where $\kappa = 0.41$ is the von K\'arm\'an constant and the damping
function is
\begin{equation}
  f(Ri_g) = \max\!\left(0,\; 1 - \frac{Ri_g}{Ri_c}\right),
  \qquad
  Ri_g = \frac{g \beta \, \partial_z C}{(\partial_z u)^2}.
  \label{eq:damping}
\end{equation}
Here $Ri_c \approx 0.25$ is the critical gradient Richardson number.
Where the local shear falls below a threshold
$|\partial_z u|^2 < \epsilon = 10^{-10}$, the Richardson number is set
to zero rather than evaluated, so that the damping function returns
$f = 1$ and $\nu_t$ is left undamped.  This avoids pathological
Richardson numbers in quiescent regions where the near-zero denominator
would otherwise produce spuriously large $Ri_g$ and incorrectly
suppress turbulent viscosity.  Alternative treatments of this singularity
exist, including phase-averaged $Ri_g$, capped $Ri_g$, and TKE-based
damping functions that decouple from instantaneous shear; the present
choice is the simplest option that preserves turbulence memory across
flow reversals in a minimal way.
The location of $\Lambda_c$ is sensitive to this choice at order-unity
level: for example, exponential damping $f(Ri_g) = \exp(-Ri_g/Ri_c)$
permits nonzero $\nu_t$ at all $Ri_g$, which would raise $\Lambda_c$
relative to the linear cutoff used here.

Because $\nu_t$ is proportional to $|\partial_z u|$, it vanishes
momentarily at every flow reversal regardless of the local
stratification.  This periodic reset of the turbulent viscosity
introduces an intrinsic intermittency in the mixing that may contribute
to the episodic laminarisation events observed near the transition
boundary (\S\,\ref{sec:nonmono}).

\subsection{Non-dimensionalization}

Using the Stokes layer thickness $\delta = \sqrt{2\nu/\omega}$ as the length
scale, $U_0$ as the velocity scale, $1/\omega$ as the time scale, and $C_0$
as the concentration scale, the system is governed by three dimensionless
groups:
\begin{equation}
  Re = \frac{U_0 \delta}{\nu}, \qquad
  S = \frac{w_s}{U_0}, \qquad
  \Lambda = \frac{g \beta C_0 \delta}{U_0^2}.
  \label{eq:params}
\end{equation}
Here $\beta = (\rho_s - \rho_f)/\rho_f$ is the submerged specific gravity
of the sediment (equivalently, $s - 1$ where $s = \rho_s/\rho_f$), so
that the buoyancy term $g\beta C$ represents the reduced-gravity
contribution from suspended sediment.

\subsection{Boundary conditions}

At the bed ($z = 0$): no-slip $u = 0$ and fixed reference concentration
$C = C_{\mathrm{ref}}$. At the domain top ($z = H$): free-slip
$\partial_z u = 0$ and zero sediment flux $\partial_z C = 0$.
The concentration scale $C_0$ used to define $\Lambda$ in
\eqref{eq:params} is identical to the bed boundary value
$C_{\mathrm{ref}}$; in the non-dimensional system both equal unity, so
$\Lambda$ directly measures the ratio of buoyancy to inertial forcing
at the bed.
