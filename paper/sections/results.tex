\section{Results}

We carried out four core phases of parameter sweeps plus three
supplementary analyses, totalling 1101 individual simulations across the
$(Re, S, \Lambda)$ parameter space.  Phase~1 performed a full
reconnaissance (270~cases at $N=256$, 20~oscillation cycles), Phase~2
verified grid convergence (24~cases at $N=32,64,128,256$), Phase~3
refined the transition brackets (377~cases at $N=256$, 20~cycles), and
Phase~4 produced publication-quality runs at the 30~cases closest to the
regime boundary ($N=256$, 40~cycles).  Phase~5 repeated the full sweep
with an exponential damping function (270~cases at $N=256$), Phase~6
computed re-entrant vertical profiles at high resolution (3~cases at
$N=256$, 40~cycles), and Phase~7 densified coverage near the re-entrant
transition edges at $S=0.005$ and $S=0.01$ (127~cases at $N=256$,
20~cycles).  Table~\ref{tab:sweep-phases} summarises each phase.

\begin{table}[htbp]
\centering
\caption{Summary of sweep phases.}
\label{tab:sweep-phases}
\begin{tabular}{clcccc}
\toprule
Phase & Purpose & $N$ & Cases & Converged \\
\midrule
1 & Reconnaissance             & 256     & 270 & 263 (97\%) \\
2 & Grid convergence           & 32--256 &  24 &  22 (92\%) \\
3 & Refined transition         & 256     & 377 & 358 (95\%) \\
4 & Production near-transition & 256     &  30 &  29 (97\%) \\
\midrule
  & Core total                 &         & 701 & 672 (96\%) \\
\midrule
5 & Exponential damping        & 256     & 270 & 267 (98\%) \\
6 & Re-entrant profiles        & 256     &   3 &   3 (100\%) \\
7 & Re-entrant expansion       & 256     & 127 & 117 (92\%) \\
\midrule
  & Grand total                &         & 1101 & 1059 (96\%) \\
\bottomrule
\end{tabular}
\end{table}

\subsection{Validation}

The solver is validated against three analytic benchmarks: (i)~convergence
of stretched-grid finite differences applied to known functions,
(ii)~implicit diffusion of a sinusoidal initial condition compared with the
analytic heat kernel, and (iii)~the Stokes oscillatory boundary layer
solution $u(z,t) = U_0 e^{-z/\delta}\sin(\omega t - z/\delta)$ run with
zero sediment coupling.

\paragraph{Turbulent baseline.}
To verify that the algebraic closure produces physically reasonable
turbulence intensities, we compare the predicted wave friction factor
$f_w = (\pi/2)\, c_f$ against the experimental measurements of
\citet{jensen1989} for clear-fluid ($\Lambda = 0$) oscillatory boundary
layers.  At $Re_\delta = 803$ ($Re_a \approx 3.2 \times 10^5$ in the
Jensen convention), the model yields $f_w \approx 0.091$, within 20\% of
the measured value $f_w = 0.114$.  At $Re_\delta = 394$ ($Re_a \approx
7.8 \times 10^4$), near the laminar--turbulent transition, the model
underestimates $f_w$ by roughly 46\% ($f_w \approx 0.121$ vs.\
$0.226$).  This discrepancy is expected: the algebraic mixing-length
closure lacks turbulent kinetic energy transport, which is particularly
important in the transitional regime where turbulence is intermittent
and phase-dependent.  For regime classification purposes---distinguishing
turbulent from laminar-like states---the order-of-magnitude agreement
is adequate, and the model correctly identifies both Reynolds numbers
as turbulent ($\langle\nu_t\rangle/\nu > 10$).

\subsection{Grid convergence}

Six representative cases spanning the laminar, turbulent, and
near-transition regimes were run at $N = 32, 64, 128,$ and $256$
(Fig.~\ref{fig:grid-convergence}).  Clearly laminar ($Re=100$) and
clearly turbulent ($Re=500$, $\Lambda=0$) cases are grid-independent:
$\langle\nu_t\rangle/\nu$ varies by less than 10\% across all
resolutions.  The threshold $\langle\nu_t\rangle/\nu = 10$ represents
approximately one order of magnitude of turbulent enhancement over
molecular viscosity, providing a physically meaningful demarcation
between laminar and turbulent states.  In the context of the momentum
equation, this threshold corresponds approximately to the onset of
order-unity Reynolds stress relative to viscous stress---i.e., the point
at which turbulent momentum transfer begins to dominate---rather than a
strict binary between turbulent and non-turbulent states.  The regime
classification is robust to moderate variations in this threshold: using
values of 8 or 12
does not alter the classification for clearly laminar or clearly
turbulent cases, although near-transition cases remain sensitive by
definition.  Phase-localized turbulence (i.e., turbulence confined to
specific phases of the oscillation cycle) was examined qualitatively in
the near-transition cases and did not alter the cycle-averaged regime
classification.  Near-transition cases, however, are sensitive to
resolution.  One of six cases changes its regime classification
between $N=32$ and $N=64$: $Re=500$, $S=0.01$, $\Lambda=0.1$
is classified as laminar at $N=32$ ($\langle\nu_t\rangle/\nu = 7.7$)
but turbulent at $N=64$ (17.3), $N=128$ (17.7), and $N=256$ (17.9).

At high Reynolds number, the turbulent viscosity ratio grows with
resolution: for $Re=1000$, $S=0.1$, $\Lambda=5.0$, the ratio increases
from 30.1 ($N=32$) to 34.5 ($N=256$), indicating that coarse grids
under-resolve turbulent transport.  Based on these results, $N=128$ is
the minimum reliable resolution for transition studies, and $N=256$ is
used for all production runs (Phases~1, 3--7).

\figGridConvergence{}

\subsection{Regime diagram}

Figure~\ref{fig:regime-diagram} presents the regime classification
across the full $(Re, S, \Lambda)$ parameter space, with one panel per
settling number $S$.  Each point represents a converged simulation from
Phases~1, 3, and~7 (666~unique converged cases at $N=256$), classified
as turbulent (red squares, $\langle\nu_t\rangle/\nu \geq 10$) or
laminar (blue circles).  Throughout this paper, ``laminar'' in the
context of regime classification denotes the laminar-like state
identified by $\langle\nu_t\rangle/\nu < 10$, which retains nonzero but
weak turbulent viscosity; it is not synonymous with the analytically
laminar Stokes solution.

\figRegimeDiagram{}

The regime map reveals three key features:

\begin{enumerate}
\item \textbf{Base transition at $Re \approx 100$--$200$.}  Below
  $Re = 200$, all cases are laminar regardless of stratification, with
  $\langle\nu_t\rangle/\nu$ between 2.2 and 8.6.  At $Re = 200$, all
  cases are weakly turbulent ($\langle\nu_t\rangle/\nu \approx 12.7$).
  Above $Re = 300$, most cases are turbulent unless sufficient
  stratification is applied.

\item \textbf{Settling number controls stratification effectiveness.}
  For slow settling ($S \leq 0.01$), sediment distributes vertically and
  stratification damps turbulence at moderate $\Lambda$.  For fast settling
  ($S = 0.5$), sediment remains confined near the bed, leaving
  turbulence undamped: all $Re = 300$ cases at $S = 0.5$ are turbulent
  ($\langle\nu_t\rangle/\nu \approx 12.1$--$12.2$ across all $\Lambda$).

\item \textbf{Non-monotonic regime boundaries.}  At $S = 0.005$ and
  $S = 0.01$, the regime classification oscillates between laminar and
  turbulent as $\Lambda$ increases for $Re \geq 300$.  For example,
  at $Re = 1000$, $S = 0.005$, the flow is turbulent for
  $\Lambda \leq 0.1$, laminar for $0.25 \leq \Lambda \leq 2.0$, and
  turbulent again at $\Lambda \geq 4.6$.  This feature persists across
  grid resolutions and integration lengths
  (see \S\,\ref{sec:nonmono}).  We note, however, that the
  re-entrant behaviour may also be amplified by the linear $Ri_g$
  damping function $f(Ri_g) = \max(0,\, 1 - Ri_g/Ri_c)$, whose sharp
  cutoff at $Ri_g = Ri_c$ can intensify feedback loops between $\nu_t$
  and $C$; a smoother closure could modify the width and sharpness of
  the re-entrant region.
\end{enumerate}

\subsection{Viscosity ratio profiles}

Figure~\ref{fig:viscosity-slices} plots the cycle-averaged viscosity
ratio against $\Lambda$ for six $(Re, S)$ slices, combining Phases~1, 3, and~7 data at $N = 256$.  The dashed line marks the classification
threshold at $\langle\nu_t\rangle/\nu = 10$.

\figViscositySlices{}

For low settling numbers ($S = 0.005$ at $Re = 300$ and $S = 0.01$ at
$Re = 500$), the viscosity ratio shows non-monotonic behaviour:
$\langle\nu_t\rangle/\nu$ drops below 10 at intermediate $\Lambda$,
then recovers at higher $\Lambda$.  At higher settling numbers ($S = 0.05$,
$S = 0.1$, $S = 0.5$), all cases remain turbulent across the full
$\Lambda$ range, as fast settling confines sediment near the bed and
prevents the outer-layer stratification that drives laminarization.

\subsection{Critical stratification}

The corrected settling flux produces a transition topology that differs
qualitatively from a simple critical $\Lambda_c$.  Rather than a sharp,
monotonic boundary between turbulent and laminar states, most $(Re, S)$
combinations exhibit non-monotonic behaviour: the flow transitions from
turbulent to laminar at a first onset $\Lambda_\ell$ and, at higher
$\Lambda$, recovers to a re-entrant turbulent state at $\Lambda_r$
(Fig.~\ref{fig:critical-lambda}).  Table~\ref{tab:lambda-c} summarises
this topology.

\begin{table}[htbp]
\centering
\caption{Transition topology from Phases~1, 3, and~7 ($N=256$) for low
settling numbers.  $\Lambda_\ell$: first laminar onset; $\Lambda_r$:
re-entrant turbulent recovery.  A dash indicates no laminar regime was
observed.  Phase~7 densified coverage near transition edges, resolving
the Re=750 case and confirming two-band structure at Re=300.}
\label{tab:lambda-c}
\begin{tabular}{cccccl}
\toprule
$Re$ & $S$ & $\Lambda_\ell$ & $\Lambda_r$ & $N_{\text{lam}}$ & Notes \\
\midrule
 300 & 0.005 & $\sim 0.029$ & $\sim 0.061$ & 15 & Two bands$^\dagger$ \\
 500 & 0.005 & $\sim 0.30$ & $\sim 1.26$  & 9 & \\
 750 & 0.005 & $\sim 0.30$ & $\sim 1.78$  & 4 & Resolved by Phase~7 \\
1000 & 0.005 & $\sim 0.28$ & $\sim 2.55$ & 4 & Wide laminar band \\
 300 & 0.01  & $\sim 0.061$ & $\sim 0.14$ & 13 & \\
 500 & 0.01  & ---         & ---         & 0 & No laminar at $N=256$ \\
 750 & 0.01  & ---         & ---         & 0 & No laminar at $N=256$ \\
1000 & 0.01  & ---         & ---         & 0 & No laminar at $N=256$ \\
\bottomrule
\multicolumn{6}{l}{\footnotesize $^\dagger$ Two disjoint $\Lambda$ intervals produce laminar states; second band: $\Lambda_\ell \sim 0.50$, $\Lambda_r \sim 0.68$.}
\end{tabular}
\end{table}

\figCriticalLambda{}

The first laminar onset is $O(0.1\text{--}2.0)$, substantially higher
than the pre-correction estimates of $O(0.01\text{--}0.1)$.  The
corrected settling flux reduces the effective stratification at a given
$\Lambda$, requiring stronger sediment loading to trigger
laminarization.  A single critical $\Lambda_c$ is ill-posed at most
$(Re, S)$ because the transition is non-monotonic: the flow laminarizes
at intermediate $\Lambda$ but recovers turbulence at higher $\Lambda$
where the concentration profile adjusts to reduce $Ri_g$
(\S\,\ref{sec:nonmono}).  The non-monotonicity is pervasive, appearing
at $S = 0.005$ and $S = 0.01$ across all Reynolds numbers tested, not
merely at $S \geq 0.1$ as in the pre-correction results.

\subsection{Production diagnostics}
\label{sec:production}

Table~\ref{tab:phase4} presents the full diagnostic output from the
30~Phase~4 production runs ($N=256$, 40~cycles), including the drag
coefficient $c_f$ and cycle-averaged kinetic energy $\langle KE\rangle$.
29 of 30 cases converged, a marked improvement over Phase~3 (67\%
convergence), attributable to the higher resolution and longer
integration.  All 30~cases are classified as turbulent; the $Re=200$ cases, previously
classified as laminar under the old viscosity ratio averaging, now yield
$\langle\nu_t\rangle/\nu \approx 12.7$ and are weakly turbulent.

\begin{table}[htbp]
\centering
\caption{Phase~4 production diagnostics ($N=256$, 40~cycles).
$c_f = \tau_{\mathrm{bed}} / (\frac{1}{2} U_0^2)$;
$\langle KE\rangle = \langle\int\frac{1}{2}u^2\,\mathrm{d}z\rangle$.
L: laminar ($\langle\nu_t\rangle/\nu < 10$);
T: turbulent ($\langle\nu_t\rangle/\nu \geq 10$).}
\label{tab:phase4}
\small
\begin{tabular}{rrlcrrr}
\toprule
$Re$ & $S$ & $\Lambda$ & Regime & $\langle\nu_t\rangle/\nu$ & $c_f$ & $\langle KE\rangle$ \\
\midrule
 300 & 0.005 & 0.25 & T & 13.7 & 0.0861 & 2.802 \\
 300 & 0.005 & 0.75 & T & 15.9 & 0.0862 & 2.677 \\
 300 & 0.005 & 1.25 & L &  9.3 & 0.0872 & 3.276 \\
 300 & 0.005 & 2.50 & T & 18.4 & 0.0864 & 2.831 \\
\midrule
 500 & 0.005 & 0.10 & T & 12.7 & 0.0696 & 3.246 \\
 500 & 0.005 & 0.50 & L &  8.8 & 0.0693 & 3.107 \\
 500 & 0.005 & 1.25 & T & 19.8 & 0.0693 & 2.538 \\
 500 & 0.005 & 2.00 & T & 25.1 & 0.0695 & 2.739 \\
 500 & 0.005 & 3.00 & T & 28.4 & 0.0697 & 2.868 \\
 500 & 0.010 & 0.25 & T & 12.4 & 0.0697 & 3.250 \\
 500 & 0.010 & 0.75 & T & 10.9 & 0.0697 & 3.201 \\
 500 & 0.010 & 1.25 & T & 22.4 & 0.0695 & 2.712 \\
 500 & 0.010 & 2.50 & T & 27.8 & 0.0696 & 2.821 \\
\midrule
 750 & 0.005 & 0.10 & T & 24.5 & 0.0583 & 2.845 \\
 750 & 0.005 & 0.50 & L &  6.6 & 0.0581 & 2.901 \\
 750 & 0.005 & 1.50 & L &  7.3 & 0.0587 & 3.047 \\
 750 & 0.005 & 3.00 & T & 15.5 & 0.0595 & 3.240 \\
 750 & 0.010 & 0.10 & T & 31.0 & 0.0584 & 2.907 \\
 750 & 0.010 & 0.50 & T & 18.5 & 0.0583 & 2.824 \\
 750 & 0.010 & 1.50 & T & 11.8 & 0.0590 & 3.175 \\
 750 & 0.010 & 3.50 & T & 18.3 & 0.0598 & 3.333 \\
\midrule
1000 & 0.005 & 0.10 & T & 19.1 & 0.0522 & 3.232 \\
1000 & 0.005 & 0.50 & L &  4.9 & 0.0517 & 2.904 \\
1000 & 0.005 & 1.50 & L &  3.7 & 0.0522 & 3.016 \\
1000 & 0.005 & 3.00 & T & 13.5 & 0.0529 & 3.161 \\
1000 & 0.005 & 5.00 & T & 20.5 & 0.0533 & 3.292 \\
1000 & 0.010 & 0.10 & T & 22.1 & 0.0525 & 3.325 \\
1000 & 0.010 & 0.50 & T & 13.7 & 0.0520 & 3.150 \\
1000 & 0.010 & 1.50 & T & 10.7 & 0.0523 & 3.107 \\
1000 & 0.010 & 3.00 & T & 16.8 & 0.0531 & 3.239 \\
\bottomrule
\end{tabular}
\end{table}

Figure~\ref{fig:production-diagnostics} shows the drag coefficient and
kinetic energy as functions of the viscosity ratio for all Phase~4
cases.

\figProductionDiagnostics{}

The drag coefficient decreases systematically with $Re$ for the
turbulent cases: $c_f \approx 0.086$ at $Re = 300$,
$c_f \approx 0.070$ at $Re = 500$, $c_f \approx 0.058$--$0.059$ at
$Re = 750$, and $c_f \approx 0.052$--$0.054$ at $Re = 1000$,
consistent with the expected scaling $c_f \sim Re^{-1/2}$ for
oscillatory boundary layers.  The weakly turbulent $Re = 200$ cases have
$c_f \approx 0.103$, higher than all other turbulent cases, consistent with
the thicker boundary layer at low $Re$.
Within each $Re$, the drag coefficient is nearly insensitive to $S$ and
$\Lambda$, confirming that near-bed shear is dominated by the oscillatory
forcing rather than the turbulent enhancement at these Reynolds numbers.
Consequently, $c_f$ is a poor discriminator of the laminarization
transition; turbulent viscosity ratio or Reynolds stress diagnostics are
more reliable indicators.

The kinetic energy spans $\langle KE\rangle = 2.07$--$3.03$ across
all cases.  The weakly turbulent $Re = 200$ cases cluster at
$\langle KE\rangle \approx 2.66$, within the overall range.  The
lowest values occur at high $Re$ and near-transition $\Lambda$ (e.g.,
$Re = 1000$, $S = 0.01$, $\Lambda = 0.25$: $\langle KE\rangle = 2.07$)
where stratification partially suppresses the velocity profile even
though the flow remains turbulent.  Cases far from the transition
boundary cluster near $\langle KE\rangle \approx 2.8$--$3.0$.

\subsection{Phase portraits}
\label{sec:phase-portraits}

Figure~\ref{fig:phase-portraits} shows the cycle-averaged phase
portraits $(u_{\mathrm{bed}},\, \tau_{\mathrm{bed}})$ for four
representative Phase~4 cases spanning the weakly and strongly turbulent regimes.

\figPhasePortraits{}

The Phase~4 set spans weakly turbulent ($Re=200$) to strongly turbulent
($Re \geq 300$) cases, and the phase portraits illustrate the spectrum
of turbulent intensities.  The weakly turbulent
$Re = 200$ cases ($\langle\nu_t\rangle/\nu \approx 12.7$) produce
tight elliptical orbits with low $\tau_{\mathrm{bed}}$ amplitude,
while the strongly turbulent case ($Re=1000$, $S=0.01$,
$\Lambda=5.0$, $\langle\nu_t\rangle/\nu = 51.4$) shows a broader
loop with substantially higher bed shear stress.  This continuous
variation in attractor geometry is consistent with the algebraic
closure model, which produces a smooth $\nu_t$ field rather than a
discrete switch between regimes.

\subsection{Non-monotonic regime behaviour}
\label{sec:nonmono}

A pervasive feature of the corrected results is non-monotonic regime
behaviour at low settling numbers ($S = 0.005$ and $S = 0.01$) across
all Reynolds numbers above the base transition.  The most striking
example is $Re = 1000$, $S = 0.005$, where the combined
Phase~1/3/7 data show: turbulent for $\Lambda \leq 0.27$
($\langle\nu_t\rangle/\nu = 10.3$--$51.6$), laminar for
$0.28 \leq \Lambda \leq 2.5$ ($\langle\nu_t\rangle/\nu = 8.9$--$9.9$),
and turbulent again at $\Lambda \geq 2.6$
($\langle\nu_t\rangle/\nu = 10.4$--$51.5$).  This wide laminar band
spanning nearly a decade in $\Lambda$ is bounded on both sides by
turbulent states.

This non-monotonicity survives across all resolutions from $N = 64$
through $N = 256$, and persists from 20 to 40~oscillation cycles.  We
interpret this as a consequence of competing effects: at intermediate
$\Lambda$, stratification damps the mixing length and $Ri_g$ exceeds
$Ri_c$ across much of the boundary layer; at high $\Lambda$, the very
strong sediment loading produces a sharp concentration gradient confined
near the bed, which reduces $Ri_g$ in the outer layer and allows
turbulence to re-establish.

Figure~\ref{fig:reentrant_profiles} shows the cycle-averaged vertical
profiles of concentration $C(z)$, gradient Richardson number $Ri_g(z)$,
and eddy viscosity $\nu_t(z)$ for three values of $\Lambda$ at
$(Re, S) = (1000, 0.005)$, computed on a fine grid ($N = 256$,
40~cycles).  At $\Lambda = 0.1$ (turbulent,
$\langle\nu_t\rangle/\nu = 19.9$), the stratification is insufficient
to push $Ri_g$ above $Ri_c$ over most of the boundary layer, and $\nu_t$
remains elevated.  At $\Lambda = 2.0$ (laminar,
$\langle\nu_t\rangle/\nu = 7.8$), the stronger sediment loading
produces a concentration profile whose gradient raises $Ri_g$ above
$Ri_c$, collapsing $\nu_t$.  At $\Lambda = 5.0$ (re-entrant turbulent,
$\langle\nu_t\rangle/\nu = 51.4$), the still-stronger sediment loading
concentrates sediment so sharply near the bed that the outer-layer
concentration gradient is actually reduced; $Ri_g$ falls below $Ri_c$ in
the outer layer, and turbulence recovers with vigour.

\figReentrantProfiles{}

These profiles confirm the proposed mechanism: the non-monotonicity arises
from the competition between stratification strength (which increases
monotonically with $\Lambda$) and the confinement of sediment near the bed
(which at high $\Lambda$ reduces the outer-layer concentration gradient
and hence $Ri_g$).  The sensitivity analysis in
\S\,\ref{sec:damping_sensitivity} further confirms that this re-entrant
behaviour is robust to the choice of damping function.
