\section{Numerical Method}

\subsection{Spatial discretization}

The vertical domain $[0, H]$ is discretized on a stretched grid using the
mapping
\begin{equation}
  z(\xi) = H \left[1 - \frac{\tanh\!\bigl(\gamma(1-\xi)\bigr)}{\tanh(\gamma)}\right],
  \qquad \xi \in [0,1],
  \label{eq:grid}
\end{equation}
where $\gamma$ controls the clustering of points near the bed. Derivatives
in physical space use second-order non-uniform finite differences applied
directly in physical $z$-coordinates, avoiding consistency errors from
chain-rule transformations through computational space. The
variable-coefficient diffusion operator
$\partial_z[\nu(z)\,\partial_z\,\cdot\,]$ is discretized using a
conservative finite-volume stencil and assembled as a tridiagonal matrix.

\subsection{Time integration}

We use an IMEX (implicit--explicit) Euler scheme. At each timestep the
turbulent viscosity $\nu_t^n$ is computed from the current fields and
frozen. The momentum equation is advanced as
\begin{equation}
  (I - \Delta t\, L_\nu)\, u^{n+1}
  = u^n + \Delta t\, F_0 \sin(\omega t^n),
  \label{eq:imex_u}
\end{equation}
where $L_\nu$ is the tridiagonal diffusion operator with effective
viscosity $\nu + \nu_t^n$. The sediment equation is advanced as
\begin{equation}
  (I - \Delta t\, L_D)\, C^{n+1}
  = C^n - \Delta t\, \partial_z(w_s\, C^n),
  \label{eq:imex_c}
\end{equation}
with $L_D$ built from $D_t = \nu_t^n / Sc_t$ and the settling flux
discretized with first-order upwind differencing. First-order upwinding
is chosen for its unconditional monotonicity, which prevents unphysical
negative concentrations and suppresses spurious oscillations near sharp
concentration gradients. The associated numerical diffusion, however,
smooths vertical concentration gradients, reducing the local gradient
Richardson number $Ri_g$ and potentially biasing the critical
stratification parameter $\Lambda_c$ toward higher values---i.e., the
model may underestimate the effectiveness of stratification in
suppressing turbulence. The net effect is a bias toward
over-stabilization: the reported $\Lambda_c$ values should therefore be
regarded as conservative upper bounds. Higher-order flux-limited schemes (e.g., TVD
methods) would reduce this diffusive bias and are a candidate for future
refinement. Both implicit systems are solved by the Thomas algorithm.

\subsection{Timestep selection}

Implicit diffusion removes the diffusive CFL restriction. The timestep is
set by the explicit settling term:
$\Delta t = \mathrm{CFL} \cdot \Delta z_{\min} / \max(w_s, U_0)$
with $\mathrm{CFL} = 0.4$.

\subsection{Cycle-averaged viscosity ratio}

The turbulent viscosity ratio used for regime classification is
$\langle\nu_t\rangle/\nu$, where $\langle\cdot\rangle$ denotes a
cycle average followed by a domain average.  The cycle average
$\bar{\nu}_t(z)$ is computed over the final oscillation period.  The
domain average is evaluated via trapezoidal integration on the
stretched grid:
\begin{equation}
  \langle\nu_t\rangle = \frac{1}{H}\int_0^H \bar{\nu}_t(z)\,\mathrm{d}z,
  \label{eq:nut_avg}
\end{equation}
which accounts for the non-uniform spacing inherent in the grid
mapping~\eqref{eq:grid}.
