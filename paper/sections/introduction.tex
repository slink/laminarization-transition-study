\section{Introduction}

Oscillatory boundary layers driven by surface waves control the transport of
sediment in coastal and continental-shelf environments. Standard engineering
models assume fully turbulent conditions, parameterizing bed shear stress
through empirical drag coefficients and eddy-viscosity closures. However,
field measurements and laboratory experiments have demonstrated that
sufficiently high suspended-sediment concentrations can stratify the boundary
layer, suppress turbulent mixing, and trigger a transition to a laminar-like
flow regime \citep{winterwerp2001,ozdemir2010,lamb2004}. This laminarization alters effective drag, changes
sediment flux scaling, and invalidates standard closures.

Despite repeated observations, and although partial regime mappings have been
reported in DNS and LES studies \citep{ozdemir2010,cantero2009}, no systematic,
continuous parameter sweep spanning the full three-parameter space has been
performed; existing DNS and LES studies provide regime classifications at
discrete parameter points but do not map the complete $(Re, S, \Lambda)$
surface. Existing studies report thresholds in terms of local
parameters, but the regime space defined by the wave Reynolds
number $Re = U_0 \delta / \nu$, the settling number $S = w_s / U_0$, and the
stratification parameter $\Lambda = g \beta C_0 \delta / U_0^2$ (the ratio of
buoyancy forces from suspended sediment to oscillatory inertial forces) has not
been swept continuously.

We address this gap computationally. By coupling the vertically resolved
momentum equation with a sediment advection--diffusion equation and an
algebraic turbulence closure damped by the gradient Richardson number, we sweep
the $(Re, S, \Lambda)$ parameter space to identify the critical surface
$\Lambda_c(Re, S)$ separating turbulent and laminar regimes.
Although motivated by coastal sediment transport, the underlying stratification-turbulence interaction is generic, and the regime boundaries identified here may inform other stratified oscillatory shear flows, subject to the closure limitations discussed below.
