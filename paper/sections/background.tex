\section{Background and Prior Work}

\subsection{Oscillatory boundary layers}

The classical theory of oscillatory boundary layers originates with
\citet{stokes1851}, who derived the exact laminar solution for flow driven by a
sinusoidally oscillating free stream above a flat plate. The solution reveals
that viscous effects are confined to a thin layer of thickness
$\delta_s = \sqrt{2\nu/\omega}$, the Stokes length, which serves as the
fundamental length scale for all subsequent work on oscillatory boundary layers.
This elegant analytical result remains the baseline against which turbulent
oscillatory flows are compared \citep{schlichting2017}.

The transition from laminar to turbulent flow in oscillatory boundary layers has
been investigated through a combination of experiment, linear stability
analysis, and direct computation. \citet{vonkerczek1974}
performed a linear stability analysis of the Stokes layer, establishing critical
Reynolds number estimates for the onset of instability. \citet{hino1976,hino1983} conducted pipe-flow experiments that identified a
sequence of flow regimes with increasing Reynolds number: disturbed-laminar,
intermittent-turbulent, and fully turbulent. \citet{akhavan1991a,akhavan1991b} carried out systematic experimental and
numerical investigations of transition in bounded oscillatory flows, providing
detailed measurements of the velocity field and Reynolds stresses through the
transition process. On the fully turbulent side, the landmark experiments of
\citet{jensen1989} measured velocity profiles and
turbulence statistics at high Reynolds numbers over smooth beds, establishing
benchmark data that have been widely used for model validation. \citet{sleath1987} extended experimental investigations to rough beds, while
\citet{jonsson1966} developed wave friction factor relationships,
and \citet{grantmadsen1979} formulated the widely used
wave--current boundary layer model. The
textbooks of \citet{fredsoe1992} and \citet{nielsen1992} synthesized much of this body of work into standard
references for coastal boundary layer dynamics.
The present study is restricted to hydraulically smooth beds; the additional
complications introduced by bed roughness---including form drag and modified
near-bed turbulence production---are beyond the present scope.

Numerical simulation has played an increasingly important role in elucidating
the physics of oscillatory boundary layers. \citet{spalart1989} performed early direct numerical simulations (DNS) of
oscillatory wall-bounded flows. \citet{vittori1998} used DNS
to study transition mechanisms in detail, while \citet{costamagna2003} identified coherent structures and their role in
turbulence production and transport. \citet{salon2007}
applied large-eddy simulation to the turbulent Stokes boundary layer, and
\citet{scandura2007} examined the structure of wall turbulence during
the cycle. More recently, \citet{ozdemir2014}
conducted DNS of smooth-walled Stokes layer transition, providing a detailed
characterisation of the intermittent turbulence regime. Experimentally,
\citet{carstensen2010,carstensen2012} documented
the formation and evolution of coherent structures and turbulent spots during
transition, and \citet{mier2021} contributed further observations
of transitional dynamics.

A common feature of nearly all of the studies cited above is that they consider
clear-fluid oscillatory boundary layers, in which density is uniform and the
turbulence dynamics are governed solely by shear. In many geophysical and
engineering applications, however, the boundary layer carries suspended sediment
whose concentration gradients introduce stable density stratification. The
resulting interaction between turbulence, sediment transport, and buoyancy
forces can fundamentally alter the transition process and even suppress
turbulence entirely---a phenomenon broadly termed laminarization. Despite its
practical importance, this coupled problem has received far less systematic
attention than its clear-fluid counterpart, motivating the present
investigation.

\subsection{Sediment stratification and turbulence suppression}

Suspended sediment introduces a stable density stratification that opposes
vertical turbulent mixing. The competition between stabilizing buoyancy and
destabilizing shear is quantified by the gradient Richardson number,
$Ri_g = -(g/\rho)(\partial\rho/\partial z) / (\partial u/\partial z)^2$.
\citet{miles1961} and \citet{howard1961} established that
$Ri_g > 1/4$ everywhere is a necessary condition for the stability of an
inviscid stratified shear flow, a result that underpins all subsequent
turbulence damping criteria. The foundational treatment of buoyancy effects in
turbulent flows is given by \citet{turner1973}; more recently, \citet{ivey2008} reviewed the subtle relationship between
stratification, turbulence, and irreversible mixing.

A well-known limitation of $Ri_g$-based closures arises in oscillatory flows,
where the mean shear $\partial u/\partial z$ passes through zero twice per
cycle. Near these flow-reversal phases the instantaneous $Ri_g$ becomes
singular or extremely large, even though the flow may remain dynamically
unstable due to residual turbulent kinetic energy and finite-amplitude
perturbations. This phase-dependent singularity complicates any closure that
relies on the local, instantaneous Richardson number and motivates caution in
interpreting cycle-phase-resolved results.

Observational and experimental evidence for sediment-induced stratification
effects has accumulated over several decades. \citet{trowbridge1994} documented fluid-mud dynamics on the Amazon continental
shelf, where high near-bed sediment concentrations created strong density
gradients that suppressed turbulent momentum transfer. \citet{friedrichs2000} showed that fine-sediment accumulation in coastal
seas is closely linked to boundary-layer stratification processes, and \citet{wright2006} reviewed gravity-driven transport on continental
shelves, emphasizing the role of wave-supported fluid muds. \citet{winterwerp2001,winterwerp2006} systematically examined stratification
effects across a wide range of sediment concentrations, from dilute
non-cohesive suspensions to dense cohesive muds, identifying concentration
thresholds beyond which turbulence is effectively quenched. \citet{lamb2004} measured the turbulent structure of high-density
suspensions formed under waves, providing direct evidence of damped Reynolds
stresses at elevated concentrations. \citet{dohmenjanssen1999,dohmenjanssen2002} documented grain-size-dependent
phase lags and transport rates in oscillatory sheet flow, highlighting the
role of vertical sediment distribution in controlling near-bed dynamics.

Direct numerical simulations have clarified the underlying mechanisms. \citet{cantero2009} simulated stratification effects in
sediment-laden turbulent channel flow, demonstrating progressive turbulence
suppression as the concentration---and hence the bulk Richardson
number---increased beyond a critical level. \citet{cantero2012}
extended this work toward universal criteria for turbulence suppression in
dilute turbidity currents. Most directly relevant to the present study, \citet{ozdemir2010} performed DNS of fine-particle-laden
oscillatory channel flow and showed that particle-induced density stratification
can fundamentally alter the flow regime, driving a transition from fully
turbulent to an intermittent or laminar-like state. \citet{ozdemir2010} observed relaminarization at bulk Richardson numbers of
$O(10^{-2}\text{--}10^{-1})$, which is broadly consistent with the critical
$\Lambda_c$ values found in the present study; however, direct quantitative
comparison is complicated by differences in the turbulence treatment (DNS
versus algebraic closure). In a broader context,
relaminarization phenomena across fluid mechanics were reviewed by \citet{narasimha1979}; more recent examples in non-sediment contexts
include the pulsatile pipe flow experiments of \citet{greenblattmoss2004} and the transient channel flow study of \citet{heseddighi2013}. \citet{floresriley2011}
used DNS to analyze turbulence collapse in the stably stratified surface layer.
\citet{balachandar2010} provided an extensive review of
turbulent dispersed multiphase flows, including two-way coupling effects
relevant to particle-laden boundary layers.

Despite this body of work, a systematic mapping of the transition from turbulent
to laminar-like conditions as a function of the governing dimensionless
groups---the Reynolds number $Re$, settling number $S$, and stratification
parameter $\Lambda$---has not been undertaken. Existing studies identify
thresholds in terms of local parameters for specific flow configurations, but no
unified phase diagram delineates the critical stratification
$\Lambda_c(Re, S)$ at which turbulence collapses in oscillatory boundary layers.
The present study addresses this gap through a comprehensive computational sweep
of the $(Re, S, \Lambda)$ parameter space.

\subsection{Turbulence closures for stratified boundary layers}

The simplest algebraic turbulence closure is the Prandtl mixing-length model
\citep{prandtl1925}, which prescribes the eddy viscosity as
$\nu_t = \kappa^2 z^2 |\partial u/\partial z|$, where $\kappa \approx 0.41$ is
the von K\'arm\'an constant and $z$ is the distance from the wall. \citet{vandriest1956} introduced an exponential damping factor to enforce the
correct near-wall scaling in the viscous sublayer. Despite its simplicity, this
algebraic closure remains widely used in boundary-layer studies where the
turbulence structure is well characterised by a single length scale. More
sophisticated approaches include the $k$--$\varepsilon$ model
\citep{launderspalding1974}, the $k$--$\omega$ model \citep{wilcox1988}, and the
hierarchy of second-moment closures developed by \citet{melloryamada1974,melloryamada1982}, the latter being especially prevalent
in geophysical and ocean modelling. \citet{rodi1987} reviewed turbulence
models for stratified flows, and \citet{pope2000} provides the
comprehensive modern reference for turbulent flow closure methods.

Stable density stratification suppresses vertical mixing, and this effect is
commonly parameterised through a damping function of the gradient Richardson
number $Ri_g$. One of the earliest such formulations was proposed by \citet{munkanderson1948}. A particularly transparent form is the linear
damping function $f(Ri_g) = \max(0,\, 1 - Ri_g/Ri_c)$, where the critical
value $Ri_c = 0.25$ follows from the Miles--Howard stability criterion
\citep{miles1961,howard1961}: stratification progressively suppresses turbulent
momentum transfer as $Ri_g$ increases, and the eddy viscosity vanishes entirely
when $Ri_g \geq Ri_c$. We note that $Ri_c = 0.25$ is adopted here as a model
parameter rather than a strict physical stability threshold, since the
Miles--Howard criterion is strictly inviscid and applies to linear
perturbations of a parallel shear flow. This simple closure captures the essential bifurcation
between turbulent and laminar-like states. Validation of
stratification--turbulence interaction has been provided by direct numerical
simulation and large-eddy simulation studies. \citet{armenio2002} and \citet{taylorsarkar2005}
examined stratified channel and open-channel flows, while \citet{taylorsarkar2008} extended this work to stratified Ekman layers.
\citet{brethouwer2007} analysed the scaling regimes of
strongly stratified turbulence, \citet{zontaetal2012} constructed a phase diagram for turbulence and internal
waves in stably stratified channel flow, and \citet{matervenayagamoorthy2014} proposed a unifying parameterisation framework
linking turbulent diffusivity to stratification strength. \citet{ivey2008} reviewed the broader relationship between
stratification, turbulence intensity, and mixing efficiency.

In the present study, we deliberately adopt the simplest closure that captures
the laminarization bifurcation: a mixing-length model with linear
Richardson-number suppression. Because our objective is regime
classification---identifying the critical surface $\Lambda_c(Re,\, S)$ that
separates turbulent and laminarized states---rather than quantitative prediction
of turbulent fluxes, an algebraic closure is sufficient. More elaborate models
($k$--$\varepsilon$, $k$--$\omega$, second-moment closures) may refine the
location of the critical surface and are natural targets for future work, but
they introduce additional model parameters and transport equations that are
unnecessary for the bifurcation analysis pursued here. The specific
implementation of the eddy-viscosity model and its coupling to the sediment
transport equations are detailed in the following section. It should be noted
that the location of $\Lambda_c$ is expected to be sensitive to the choice of
damping function. Alternative forms---such as exponential damping
$f(Ri_g) = \exp(-Ri_g/Ri_c)$ or quadratic suppression---would produce smoother
or more gradual turbulence attenuation and likely shift the critical surface
quantitatively, though the qualitative bifurcation between turbulent and
laminar-like states should persist.
